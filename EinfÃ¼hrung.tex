\section{Einleitung}

\glqq die künstliche Intelligenz macht rasante Fortschritte und beeinflusst immer mehr Bereiche unseres Lebens.\grqq \cite{2}


\subsection{Problemstellung}
Ein bekanntes Verfahren der KI ist das Deep Learning. Deep Learning ist ein vielversprechender Ansatz. Es findet im Bereich des überwachten, sowie unüberwachten maschinellem Lernens große Anerkennung. Ein interessantes Modell in letzterem Bereich ist das Generativ Adversarial Network, kurz GAN. Dieses wurde 2014 von Ian Goodfellow vorgestellt. 
So stark (überzeugend) dieses Verfahren auch ist, sieht es sich dennoch einigen Problemen gegenübergestellt. 

\glqq beruhe der Entwurf neuronaler Netze auf einer großen Willkür: Bei der Entscheidung, wie viele Lagen mit wie vielen Neuronen genutzt werden sollten, beruhe vieles auf Bauchgefühl oder auf Ausprobieren.\grqq{} \cite[S.44]{3}

\subsection{Zielsetzung}
Diese Arbeit wird neuartige Netzwerkarchitekturen für die Bildgenerierung vorstellen. Basis bilden hierbei bekannte Architekturen. Diese werden ausgewertet ob und wie weit Generative Adversarial Networks dadurch effizienter trainiert werden können. Das Ziel erfolgreichen Trainings ist die Generierung täuschend echter Bilder.  