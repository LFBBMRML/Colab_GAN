\documentclass[12pt, bibliography=totoc, listof=totoc, listof=entryprefix]{scrartcl}

%Usepackages
\usepackage[german]{babel}
\usepackage[utf8]{inputenc}
\usepackage[T1]{fontenc}
\usepackage{color}
\usepackage{amssymb}
\usepackage{amsthm}
\usepackage{amsmath}
\usepackage{graphicx}
\usepackage{wrapfig}
%\usepackage[automark]{scrpage2}
\usepackage[automark]{scrlayer-scrpage}
\usepackage{color}
\usepackage{array}
\usepackage{standalone}
\usepackage{tabularx}
%\usepackage{tocloft}
\usepackage{url}
\usepackage{longtable}
\usepackage{hyperref}
\usepackage{chngcntr}
\usepackage{multirow}
\usepackage{enumitem}
%\usepackage{pdflscape}
\usepackage{nccmath}
\usepackage{acronym}
\usepackage{textcomp}
\usepackage{booktabs}
\usepackage{pdflscape}
\usepackage{listings}
\lstset{language=Pascal}
\renewcommand{\lstlistingname}{Code}
\usepackage{bytefield}
\usepackage[onehalfspacing]{setspace}
\usepackage{colortbl}
\usepackage{longtable}
\usepackage{subfigure}
\usepackage{tikz}
\usepackage{pgfplots}
\pgfplotsset{compat=1.10}
\usepackage{caption}

%Formale Gestaltung nach Studienamt Vorschlag
%Serifenschrift (Ähnelt Times New Roman)
\usepackage{times}
%\usepackage{helvet}
%Schriftgröße 12 s.o
%Zeilenabstand
\renewcommand{\baselinestretch}{1.30}
\usepackage[a4paper,lmargin={3.5cm},rmargin={1.5cm},tmargin={3.0cm},bmargin = {2.5cm}]{geometry}

%Bibliothek
\usepackage[backend=bibtex, style=numeric]{biblatex}
\usepackage[babel,german=quotes]{csquotes}

\addbibresource{Ramona_Literatur.bib}

%Grafiken
\usepackage[]{epstopdf}
\usepackage{float}
%\usepackage{gnuplottex}

%Kopfzeile
\pagestyle{scrheadings}
\clearscrheadfoot 
\ihead[\headmark]{\headmark}
\ofoot[\pagemark]{\pagemark}
\automark[section]{section}
\setheadsepline{0.4pt}

\headsep15mm

%Zeilenhöhe Tabelle
\renewcommand{\arraystretch}{1}

%nummerierung Abbildungen und Tabellen
\numberwithin{equation}{section}
\parindent0pt 
\renewcommand{\baselinestretch}{1.3}
\renewcommand{\thefigure}{\arabic{section}.\arabic{figure}}
\renewcommand{\thetable}{\thesection.\arabic{table}} 
\counterwithin{figure}{section}
\counterwithin{table}{section}


%Nummerierung Inhaltsverzeichnis
\setcounter{secnumdepth}{4}
\setcounter{tocdepth}{3}
\addtolength{\footskip}{-1cm}
%Für Formelverzeichnis
\DeclareNewTOC[%
 type=formel,
 name={Formel},%
 hang=5em,%
 listname={Formelverzeichnis}
]{for}
\newcommand*{\formelentry}[1]{%
 \addcontentsline{for}{formel}{\protect\numberline{\theequation} #1}%
}


\begin{document}
	

%\documentclass[standalone]

\thispagestyle{empty}
%\begin{document}


	\begin{titlepage}
	
			%\begin{tabular} {p{\textwidth}}
				
				
				
				\begin{flushright}	
					\includegraphics[scale=0.35]{img/hs.jpg}
			
					\Large{\textbf{{Hochschule Kempten\\}}}
					\footnotesize{University of Applied Sciences \\}
				\end{flushright}
				
				
				
				~\\
				
				\begin{center}
					\LARGE{MASTERTHESIS}
				\end{center}
				~\\
				\begin{center}
					\textbf{\large{Entwicklung und Evaluation neuartiger neuronaler Netzwerkarchitekturen für die Bildgenerierung im Bereich des unüberwachten Lernens\\}}
				\end{center}
				
				~\\
				
				\begin{center}
					vorgelegt von
				\end{center}
				
				\begin{center}
					\large{\textbf{Ramona Judith Neher}} \\
					\large{Matrikelnummer 313747}
				\end{center}
				
				
				\begin{center}
					am\\
					\large{01. März 2021}
				\end{center}
				
				\hspace{15cm}
				
				\begin{center}
					\begin{tabular}{lll}
						\textbf{Aufgabensteller:} & & \hspace{4cm} Prof. Dr.-Ing. Jürgen Brauer\\
						
						\textbf{Betreuer:} & & \hspace{4cm} Bonifaz Stuhr\\
						[10pt]
						\textbf{durchgeführt in der:} & & \hspace{4cm} Fakultät Informatik \\[10pt]
					\end{tabular}
				\end{center}
				
			%\end{tabular}

	\end{titlepage}
%\end{document}


\newpage

%Seitennummerierung Römisch
\pagenumbering{Roman}
\setcounter{page}{1}

\label{Sperrvermerk}

%\begin{document}

%\begin{bfseries}

%\[%keine Kopfzeile und Nummerierung
\section*{Sperrvermerk}
\markboth{\MakeUppercase {Sperrvermerk}}{}

Veröffentlichungen oder Vervielfältigungen der nachfolgenden Arbeit - auch nur auszugsweise oder in elektronischer Form - sind ohne ausdrückliche schriftliche Genehmigung nicht gestattet.
Hiervon ausgenommen ist eine Veröffentlichung der Kurzzusammenfassung (Abstract) der Arbeit (im Wortlaut wie sie in der Arbeit enthalten ist) - mit Angabe des Namens des Studierenden und des Titels der Arbeit - auf gedruckten Medien oder auf einer Internetseite (z.B. im Rahmen einer Auflistung der durch den Dozenten betreuten studentischen Abschlussarbeiten).

Die Sperrfrist gilt bis zum 02. Februar 2026.
Die Arbeit darf bis zum Ablauf der Sperrfrist nur für Prüfungszwecke verwendet werden.
 
%\]
%\end{bfseries}

%\end{document}
\addcontentsline{toc}{section}{Sperrvermerk}
\newpage

\section*{Abstract}

Die vorliegende Masterthesis befasst sich mit der \textit{Entwicklung und Evaluation neuartiger neuronaler Netzwerkarchitekturen für die Bildgenerierung im Bereich des unüberwachten Lernens}.

%Kurzbeschreibung, was in den einzelnen Abschnitten behandelt wird, ohne sachlich darauf einzugehen






\addcontentsline{toc}{section}{Abstract}
\newpage

\section*{Vorwort}

Diese Arbeit behandelt die \glqq Entwicklung und prototypische Umsetzung eines Konzeptes für ein ePlanungsboard zur verbrauchsgesteuerten Produktionsplanung\grqq. Sie wurde im Rahmen der Masterthesis meines Studiums an der Hochschule für angewandte Wissenschaften Kempten erstellt.

Die Idee zu dem Thema dieser Arbeit entstand in Zusammenarbeit mit Herr Stuhr. 

Welche Hardware habe ich genutzt, welchen Datensatz?

Ich durfte einige theoretische Grundlagen meines Studiums anwenden, die ich durch die Module. 

Ein besonderer Dank gilt meinem Betreuer Bonifaz Stur, der sich nicht nur immer Zeit genommen hat, mich zu unterstützen, sondern es als persönliches Anliegen sah, mich an mein Ziel zu führen. 
Auch meinem betreuenden Professor der Hochschule Kempten, Prof. Dr. Jürgen Brauer, möchte ich meinen Dank aussprechen. Auf eine zeitnahe und hilfreiche Antwort auf meine Fragen war immer Verlass. 
%Neben diesen beiden Personen bin ich auch meinem Chef, Dr. Arnd Kolleck, dankbar über das mir entgegengebrachte Vertrauen und die Möglichkeit zur Anfertigung meiner Arbeit in seinem Team.

Weiter möchte ich allen Personen danken, die mir bei Fragen und Problemen immer mit einer Antwort zur Seite standen und allen Korrekturlesern für ihren Aufwand und die Hilfestellung.


\addcontentsline{toc}{section}{Vorwort}
\newpage

\addcontentsline{toc}{section}{Inhaltsverzeichnis}
\tableofcontents
\newpage

\label{Abkürzungsverzeichnis}
\documentclass[20pt]{article}
\usepackage[utf8]{inputenc} % Direkte Eingabe von Umlauten und anderen Diakritika

\begin{document}

%\[%keine Kopfzeile und Nummerierung
\section*{Abkürzungsverzeichnis}
\markboth{\MakeUppercase {Abkürzungsverzeichnis}}{}

\begin{acronym} [MAXSHORT] %Auf wie viele Zeichen wird Aufgefüllt
	\acro {AdaGrad}{Adaptive Gradient}
	\acro {Adam}{Adaptive Moment Estimation}
	\acro {CNN}{Convolutional Neural Networks}
	\acro {D}{Discriminator}
	\acro {DCGAN}{Deep Convolutional Generative Adversarial Network}
	\acro {G}{Generator}
	\acro {GAN}{Generative Adversarial Network}
	\acro {KNN}{Künstliches Neuronales Netz}
	\acro {MLP}{Multilayer Perceptron}
	\acro {MSE}{Mean Squared Error}
	\acro {NN}{Neuronales Netz}
	\acro {ReLU}{Rectified Linear Unit}
	\acro {RMS}{Root Mean Square Propagation}
	\acro {tanh}{Tangens hyperbolicus}
\end{acronym}

%\]

\end{document}
\addcontentsline{toc}{section}{Abkürzungsverzeichnis}
\newpage

%Seitennummerierung arabisch
\pagenumbering{arabic}
\setcounter{page}{1}

\section{Einleitung}

\glqq die künstliche Intelligenz macht rasante Fortschritte und beeinflusst immer mehr Bereiche unseres Lebens.\grqq \cite{2}


\subsection{Problemstellung}
Ein bekanntes Verfahren der KI ist das Deep Learning. Deep Learning ist ein vielversprechender Ansatz. Es findet im Bereich des überwachten, sowie unüberwachten maschinellem Lernens große Anerkennung. Ein interessantes Modell in letzterem Bereich ist das Generativ Adversarial Network, kurz GAN. Dieses wurde 2014 von Ian Goodfellow vorgestellt. 
So stark (überzeugend) dieses Verfahren auch ist, sieht es sich dennoch einigen Problemen gegenübergestellt. 

\glqq beruhe der Entwurf neuronaler Netze auf einer großen Willkür: Bei der Entscheidung, wie viele Lagen mit wie vielen Neuronen genutzt werden sollten, beruhe vieles auf Bauchgefühl oder auf Ausprobieren.\grqq{} \cite[S.44]{3}

\subsection{Zielsetzung}
Diese Arbeit wird neuartige Netzwerkarchitekturen für die Bildgenerierung vorstellen. Basis bilden hierbei bekannte Architekturen. Diese werden ausgewertet ob und wie weit Generative Adversarial Networks dadurch effizienter trainiert werden können. Das Ziel erfolgreichen Trainings ist die Generierung täuschend echter Bilder.  
\newpage

\section{Related Work}

Was haben andere bisher publiziert? Worauf beziehe ich mich
\newpage

\input{künstliche neuronale Netze}
\newpage

\input{GAN Funktionsweise}
\newpage

\section{GAN - Generative Adversarial Network} \label{GAN}
\begin{document}

\glq \glqq the coolest idea in deep learning in the last 20 years\grqq{} \grq{} \cite{0}
So beschrieb Yann LeCun, Leiter der Facebook KI-Abteilung in New York, das von Ian J. Goodfellow in 2014 vorgestellte Verfahren gegenüber der Carnegie Mellon University in Pittsburgh, Pennsylvania. \cite[vgl.][]{0} \\


Es handelt sich um ein neuartiges \textit{deep neural network} im Bereich des unüberwachten Lernens - das \textit{Generative Adversarial Network}. Dabei stehen sich zwei konkurrierende Modelle gegenüber:
Auf der einen Seite der \textit{Generator} (engl. generator), ein \textit{generative} Modell, welches gefälschte Daten erstellt. Auf der anderen Seite der \textit{Diskriminator} (engl. discriminator), ein \textit{diskriminatives} Modell. Dessen Sinn ist es, die generierten Daten von realen Beispielen unterscheiden zu können und dient demnach als Klassifikator.
Während sich die Modelle gegenseitig trainieren, verfolgt das \textit{generative} Modell das Ziel, das \textit{diskriminative} Modell zu täuschen, sodass dieses die generierte Daten nicht von echten Beispielen differenzieren kann.
Die Zielsetzung eines GANs ist es, täuschend echte Daten maschinell zu erstellen.
Ian J. Goodfellow präsentierte die Netzwerke im Paper \cite{4} als \textit{multilayer Perzepronen}, jedoch bestehen inzwischen einige weitere Architekturen, wobei das \textit{Deep Convolutional Generative Adversarial Network} (DCGAN) an beachtlicher Bedeutung gewonnen hat.
\cites[vgl.][]{4}[S. 5]{5}[S. 7]{6}

(Generator erstellt Inferenz)
\subsection{Generative und diskriminative Modelle}
Wie bereits angeführt, setzt sich ein GAN aus einem \textit{generativen} und einem \textit{diskriminativen} Modell zusammen. Doch welchem Sinn und Zweck dienen diese Modelle konkret? Diese Frage soll nachstehend ausführlich beantwortet werden. \\

Obwohl beide Vorgehen mit echten und generierten Daten arbeiten, verfolgen sie unterschiedliche Absichten. 
Sei X eine Dateninstanz und Y zugehörige Label, so ergibt sich Tabelle \ref{tab:gen_dis_Modell}:

\begin{table}[h]
\centering

	\begin{tabular}[h]{p{2.8cm}|p{7.5cm}|p{4.5cm}}
		Modell & Wahrscheinlichkeit & Ziel 
		\rule{0pt}{1.5em}\\
		\hline
		\rule{0pt}{1.5em}
		\textit{Generatives} & bestimmt die Wahrscheinlichkeit P(X,Y), bzw. P(X), also Wahrscheinlichkeit eines konkreten Beispiels X & Erstellen von überzeugend realistischen Daten\\
		\hline
		\rule{0pt}{1.5em}
		\textit{Diskriminatives} & arbeitet mit der bedingte Wahrscheinlichkeit P(X|Y), also wie wahrscheinlich es ist, dass eine gewisse Instanz einem bestimmten Label zugeordnet werden kann & Differenzierung verschiedener Instanzen\\
	\end{tabular}
	\captionsetup{justification=centering}
	\caption{Aufgabe eines \textit{generativen} und \textit{diskriminativen} Modells \\ Daten in Anlehnung an \cite[vgl.][]{30}}
	\label{tab:gen_dis_Modell}
\end{table}

Das \textit{generative} Modell modelliert hierfür die Verteilung der Daten, die nachgebildet werden sollen, in einem Datenraum $\mathbb{R}^n$. Die Aufgabe ist nun, Instanzen zu erstellen, die möglichst dicht an den realen Werten liegen. Je geringer der Abstand zwischen den Daten, desto realistischer die Nachbildung.
Das \textit{diskriminative} Modell muss dabei eine Grenze im Datenraum bestimmen, welche Instanzen mit verschiedenen Bezeichnungen (engl. Label) voneinander separiert. Kann der Algorithmus eine optimale Abgrenzung dezidieren, können Daten zugeordnet werden, ohne Angabe der jeweiligen bedingten Wahrscheinlichkeiten. Es fungiert demnach als Klassifikator.\\

In Abbildung \ref{fig:gen_dis_Modell} bilden grüne Symbole echte und orangene generierte Daten ab. Dabei ist zu erkennen wie das \textit{generative} Modell die Verteilung modelliert und Instanzen bereits nahe der originalen Daten erstellt. Das \textit{diskriminative} Modell findet hier dennoch eine nahezu optimale Lösung. Lediglich ein Wert wurde fälschlicherweise als Echt klassifiziert.

\begin{figure}[H]
	\centering
	\begin{center}	
		\text{Generatives Modell  \hspace{3.7cm} Diskriminatives Modell}\\
		\vspace{0.5cm}
		\includegraphics[width=0.25\textwidth]{img/generatives_Modell.png}
		\hspace{3cm}
		\includegraphics[width=0.25\textwidth]{img/diskriminatives_Modell.png}
	\end{center}
	\caption{Generatives und diskriminatives Modell}
	\label{fig:gen_dis_Modell}
\end{figure}

Die Aufgabe eines \textit{generativen} Modells ist damit deutlich komplexer. Auch Ian J. Goodfellow verweist in seiner Publikation auf Schwierigkeiten der Berechnung dieses Modells. 

Ein \textit{Generative Adversarial Network} realisiert die großartige Idee, die beiden Modell konkurrieren zu lassen, wodurch ein gegenseitiges Trainieren entsteht. Davon profitiert primär das \textit{generative} Modell, denn das Feedback des \textit{diskriminativen} dient als Hilfestellung für die Modellierung neuer Daten.
So kann die Leistung ganz ohne menschlichen Eingriff sukzessive gesteigert werden, bis Instanzen entstehen, welche nicht länger von den Originalwerten differenziert werden können. 

\cite[vgl.][]{30}

\subsection{Definition eines Autoencoders und dessen Rolle in einem GAN}

Ein \textit{Autoencoder} ist ein \textit{unüberwachtes künstliches neuronales Netz}, welches Eingangsinformationen $x$ zu $z$ komprimiert, um daraus eine Nachbildung $x*$ zu erstellen.
Während $x$ und $x*$ dieselbe Dimension aufweisen, findet bei der Komprimierung zu $z$ eine Dimensionsreduktion statt, weshalb die Ausmaße von $z$ wesentlich geringer sind. \\

Da es sich hier um zwei Verfahren handeln, setzt sich dieses \textit{künstliche neuronale Netz} aus zwei Abschnitten zusammen: Der \textit{Kodierer} (engl. Encoder) und der \textit{Dekodierer} (engl. Decoder). 

Ersterer komprimiert die Informationen $x$. Die Aufgabe der Komprimierung ist es, relevante Informationen zu selektieren und nur diese weiterzuleiten. Dabei entsteht die Dimension $z$, welches damit eine Repräsentation verdichteter Daten ist. Dies befindet sich in einem sogenannten \textit{verborgenem Raum}, der in der KI-Welt als \textit{latent space} bekannt ist. $z$ beinhaltet demnach nur die wichtigsten Merkmale der Eingangsdaten. \cite{33}

Zweiterer nutzt die Werte des \textit{latent space}, um die Rekonstruktion $x*$ zu bilden. \\


So wird erreicht, dass nur wesentliche Informationen weitergeleitet werden.

Folgende Abbildung zeigt schematisch das Verfahren eines \textit{Autoencoders}. 

\begin{figure}[H]
	\centering
	\includegraphics[width=0.5\textwidth]{img/autoencoder.png}
	\caption{Autoencoder, eigene Darstellung in Anlehnung an \cite[S. 20]{5}}
	\label{fig:autoencoder}
\end{figure}

Dieses \textit{künstliche neuronale Netz} bringt einige Vorteile mit sich:

\begin{itemize}
	\item Da es sich um ein \textit{Netz} im Bereich des \textit{unüberwachten Lernens} handelt, muss im Vorfeld nicht abgeklärt werden, zu welchem Sinn und Zweck das Verfahren explizit genutzt wird. Das Netz lernt selbst anhand der Inputdaten und passt sich diesen an.
	\item Dadurch ist eine konkrete Kennzeichnung der Eingabedaten überflüssig.
	\item Es müssen lediglich die komprimierten Daten übertragen werden. Dadurch wird der Informationsfluss erhöht, trotz Verringerung des Datenumfangs.
	\item Eingangsdaten können ohne Rauschen reproduziert werden.
\end{itemize}

\glqq For generation, [...] cut off the encoder part and use only the latent space and the decoder.\grqq{} \cite[S. 24]{5}
Um bei dieser Methode dennoch eine realitätsnahe Nachbildung ermöglichen zu können, muss der \textit{latent space} näher betrachtet werden. In der Folge, dass es sich nicht länger um eine Komprimierung der Daten $x$ handelt - eine logische Konsequenz, da auf die Verwendung des \textit{Encoders} verzichtet wird - müssen ein adäquater Raum gewählt werden. Hierfür hat sich eine Gleichverteilung mit einer festen Standardabweichung und angepasstem Mittelwert bewährt, wie beispielsweise die Gauß-Verteilung.

Ein \textit{Generatives adversarial Netz} macht sich dieses Verfahren zu nutze und adaptiert das Prinzip des \textit{Decoders}, der Werte aus einem \textit{latent space} entnimmt. So entsteht der \textit{Generator} eines GANs. 
\cites[vgl.][]{31,32}[S. 18-25]{5}

\subsection{Struktur eines GANs}

Formal gesprochen besteht ein \textit{Generative adversarial Network} aus zwei differenzierbaren Funktionen mit individueller Fehlerfunktion, die durch \textit{künstliche neuronale Netze} mit zugehörigen Modellparametern $\Theta$ repräsentiert werden. Damit sind die Netzwerke mittels \textit{Backpropagation} zu optimieren. \\
Während der \textit{Diskriminator} die Fehlinterpretation von echten und gefälschten Daten minimieren möchte, versucht der \textit{Generator} die Interpretation des \textit{Diskriminator} in Bezug auf die gefälschten Beispiele zu maximieren. 
\cite[vgl.][S. 37]{5} \cite[vgl.][S. 2]{4}

\begin{figure}[H]
	\centering
	\includegraphics[width=0.8\textwidth]{img/Architektur_GAN.png}
	\caption{Architektur eines GANs, eigene Darstellung in Anlehnung an \cite[S. 37]{5}}
	\label{fig:architektur}
\end{figure}

Das Zusammenspiel der einzelnen Komponenten ist in Abbildung \ref{fig:architektur} zusammengefasst. Um die Architektur zu verstehen, werden nachfolgend die einzelnen Bestandteile detailliert betrachtet:

\begin{enumerate}
	\item Im Falle eines GANs handelt es sich bei dem \textbf{latent space} typischerweise um eine Gauß-Verteilung mit einer Standardabweichung von eins und einem Mittel bei 0. Diese enthält zu Beginn jedoch keine ausschlaggebenden Informationen. 
	Werte, die aus dieser Verteilung entnommenen werden, ergeben einen \textit{random noise vector} $z$ -  vorwiegend ein Vektor mit 100 Einträgen - welcher als Input des \textit{Generators} dient.
	\cite[vgl.][S. 18, S. 37]{5} \cite[vgl.][]{34}
	
	\item Das Ziel des \textbf{Generators} ist es, eine Datenverteilung $p\textsubscript{g}$ zu lernen, die mit der Verteilung der Trainingsdaten $p\textsubscript{x}$ konvergiert. Die Funktion $G$, die durch den \textit{Generator} repräsentiert wird, wird durch die Modellparameter $\Theta\textsubscript{g}$ bestimmt. Diese erhält als Input einen \textit{random noise vector} $z$. Wird die Funktion $G$ nun auf $z$ angewandt, entsteht ein generierter Output $G(z)$. 
	Während des Trainings lernt der \textit{Generator}, Werte die einen bestimmten Output ergeben, in dem \textit{latent space} zuzuordnen \cite{34}.\cite[vgl.][S.5, S. 37]{5}
	
	\item Die Beispiele eines \textbf{Datensatzes}, mit welchem das \textit{Generative adversarial Network} trainiert wird, sind diese, deren Verteilungen rekonstruiert werden sollen. Damit bestimmt ein Datensatz, welche Verteilung gelernt wird. \cite[vgl.][S. 7]{5}

	\item Der \textbf{Diskriminator} nimmt als Input ein echtes $x$ oder ein gefälschtes $G(z)$ Beispiel entgegen. Die Aufgabe besteht darin, zu lernen, diese richtig zu klassifizieren. Dabei gibt die Funktion $D$, welche durch den \textit{Diskriminator} vertreten und durch die Modellparameter $\Theta\textsubscript{d}$ bestimmt ist, einen Skalar $D(x)$ aus. Dieser gibt für jeden Input die geschätzte Wahrscheinlichkeit (Wert zwischen [0,1]), dass $x$ ein echtes Beispiel ist.  \cite[vgl.][S. 1f.]{4}

	\item Die \textbf{Fehlerrückführung} basiert auf der Ausgabe $D(x)$ des \textit{Diskriminators}. Dabei variiert die Notation: $D(x)$ wird verwendet, wenn es sich bei einem Input um ein echtes, $D(x^*)$ oder $D(G(z))$, wobei $D(x^*) = D(G(z))$ ist, wenn es sich um generiertes Beispiel handelt. Werden die genannten Ziele der \textit{Netze} betrachtet, ergibt sich daraus folgende Tabelle:
	
	\begin{table}[H]
		\centering
		\begin{tabulary}{20cm}{L|L|L}
			\multirow{2}*{Input} & \multicolumn{2}{c}{zu erreichende Ausgabe} 
			\rule{0pt}{1.5em}\\
				
			\rule{0pt}{1.5em}
			&  Diskriminator  &  Generator \\
			\hline
			\rule{0pt}{1.5em}
			\textbf{Echt} ($x$) & 1 & / \\
			\rule{0pt}{1.5em}
			\textbf{Generiert} ($G(z)$) & 0 & 1 
			\rule{0pt}{1.5em}
		\end{tabulary}
		\captionsetup{justification=centering}
		\caption{Der Wert der Ausgabe des Diskriminators, die das jeweilige Netz versucht zu 		erreichen}
		\label{tab:Ziel_Ausgabe}
	\end{table}
	
	damit ergibt sich folgende Terminologie, welche Tabelle \ref{tab:Dis_Output} vorlegt:\\
	\textit{True positive}: Echte Beispiele korrekt als solche klassifiziert \\
	\textit{False negative}: Echt Beispiele fälschlicherweise als generierte klassifiziert \\
	\textit{True negative}: Generierte Beispiele korrekt als solche klassifiziert \\
	\textit{False positive}: Generierte Beispiele fälschlicherweise als echte klassifiziert \\
	\cite[vgl.][S. 41]{5}
	
	
	\begin{table}[H]
		\centering
		\begin{tabulary}{20cm}{L|L|L}
			\multirow{2}*{Input} & \multicolumn{2}{c}{Ausgabe des Diskriminators} 
			\rule{0pt}{1.5em}\\
			
			\rule{0pt}{1.5em}
			&  \textasciitilde 1  &  \textasciitilde 0 \\
			\hline
			\rule{0pt}{1.5em}
			\textbf{Echt} ($x$) & True positive & False negative \\
			\rule{0pt}{1.5em}
			\textbf{Generiert} ($G(z)$) & False positive & True negative 
			\rule{0pt}{1.5em}
		\end{tabulary}
		\captionsetup{justification=centering}
		\caption{Ausgabe eines Diskriminators, Daten in Anlehnung an \cite[vgl.][S. 41]{5}}
		\label{tab:Dis_Output}
	\end{table}

\end{enumerate}

	%hier ID 5 S. 37- 41
\subsection{Der Lernprozess eines GANs}

	\subsubsection{Die Fehlerfunktionen} \label{loss-function}
	
	\subsubsection{Sequenz des Generators}
	
	\subsubsection{Sequenz des Diskriminators}
	
	\subsubsection{Herausforderungen des Lernprozesses eines GANS}

\subsection{Herausforderungen des Lernprozesses}
%auch hier zeigen sich dieselben Probleme wie in Kapitel \ref{HerausforderungenDesLernens}. Doch hier kommen noch spezielle hinzu:
\textbf{konvergiert nicht}
\textbf{Mode collapse}
%Auch die Wahl der HP muss hier nochmal eingehend betrachtet werden. Ein netz kann unterschiedlich groß sein, die Batchgröße ist wichtig, nehm ich einen Bias hinzu...?
\textbf{Wahl der Hyperparameter}
\subsection{Varianten des GANs}
%Die kleinen wie Image-to-Image and so on
%BigGAN
%WassersteinGAN
\subsection{Das Deep Convolutional Generative Adversarial Network}
%Ist auch eine Variante, aber der Code, auf welchem diese Arbeit beruht, ist ein DCGAN. Daher soll zunächst die Theorie diskutiert werden, um anschließend Codeentscheidungen zu belegen
\end{document}
\newpage

\section{Neuartige Netzwerkarchitekturen}

Ziel der neuartigen Architektur: besseres/einfacheres Training. Weniger Probleme, effizientere Bildgenerierung
\subsection{Die Idee}


\subsection{Die Architektur}
\subsubsection{Architektur1}
\subsubsection{Architektur2}

\subsection{Umsetzung der Architektur}
\subsubsection{Architektur1}
\subsubsection{Architektur2}

\subsection{Probleme?}

\newpage

\section{Evaluation der neuartigen Netzwerkarchitekturen}

\subsection{Ergebnisse}

\subsection{Probleme}


\newpage

\section{Fazit und Ausblick}

Das entstandene Vorgehen ist ziemlich krass
\newpage

\pagenumbering{Roman}
\setcounter{page}{7}


\begin{document}
	

\section{Anhang}
%\addcontentsline{toc}{section}{Anhang}
%\markboth{\MakeUppercase {Anhang}}{}

% \section*{Abbildungsverzeichnis}

\listoffigures




%\addcontentsline{toc}{subsection}{Abbildungsverzeichnis}
%\listoffigures
%\newpage

% \listoftables
%\addcontentsline{toc}{subsection}{Tabellenverzeichnis}
%\listoftables
%\newpage	

%\begin{figure}



\subsection{Abbildungsverzeichnis}
%\makeatletter
%\@starttoc{lof}% Print List of Figures
%\makeatother
%\addcontentsline{toc}{subsection}{3.1 Abbildungsverzeichnis}
\listoffigures
\newpage

\subsection{Tabellenverzeichnis}
%\makeatletter
%\@starttoc{lot}% Print List of Tables
%\makeatother 
%\addcontentsline{toc}{subsection}{Tabellenverzeichnis}
\listoftables
\newpage

%\printbibliography[notkeyword=Quelle,title={\subsection*{Literaturverzeichnis}},heading=subbibliography]
\subsection{Literaturverzeichnis}
%\addcontentsline{toc}{subsection}{Literaturverzeichnis}
\printbibliography[heading=none]
%\clearpage


%\subsection*{Erklärung und Ermächtigung}
%%\section{Erklärung und Ermächtigung}
%\label{Erklärung und Ermächtigung}

%\begin{document}


\subsection{Erklärung}



Ich versichere, dass ich diese Abschlussarbeit selbstständig angefertigt, nicht anderweitig für Prüfungszwecke vorgelegt, alle benutzten Quellen und Hilfsmittel angegeben, sowie wörtliche und sinngemäße Zitate gekennzeichnet habe.
\\[2.5cm]

Kempten, den:  \hrulefill\enspace\enspace\enspace  Unterschrift:  \hrulefill
\\[2.5cm]

%\end{document}

\subsection{Ermächtigung}
\label{Ermächtigung}
%\begin{document}

Hiermit ermächtige ich die Hochschule Kempten und den betreuenden Dozenten, Herrn Prof. Dr. Jürgen Brauer, zur Veröffentlichung der Kurzzusammenfassung (Abstract) meiner Arbeit (im Wortlaut wie sie in der Arbeit enthalten ist) - einschließlich der Angabe meines Namens, des Titels der Arbeit und der Name der Firma, bei der die Arbeit durchgeführt wurde - auf gedruckten Medien oder auf einer Internetseite (z.B. im Rahmen einer Auflistung der durch den Dozenten betreuten studentischen Abschlussarbeiten).
\\[2.5cm]

Kempten, den:  \hrulefill\enspace\enspace  Unterschrift:  \hrulefill

%\end{document}
%\addcontentsline{toc}{subsection}{Erklärung und Ermächtigung}
\newpage
\subsection{Erklärung}



Ich versichere, dass ich diese Abschlussarbeit selbstständig angefertigt, nicht anderweitig für Prüfungszwecke vorgelegt, alle benutzten Quellen und Hilfsmittel angegeben, sowie wörtliche und sinngemäße Zitate gekennzeichnet habe.
\\[2.5cm]

Kempten, den:  \hrulefill\enspace\enspace\enspace  Unterschrift:  \hrulefill
\\[2.5cm]

%\end{document}

\subsection{Ermächtigung}
\label{Ermächtigung}
%\begin{document}

Hiermit ermächtige ich die Hochschule Kempten und den betreuenden Dozenten, Herrn Prof. Dr. Jürgen Brauer, zur Veröffentlichung der Kurzzusammenfassung (Abstract) meiner Arbeit (im Wortlaut wie sie in der Arbeit enthalten ist) - einschließlich der Angabe meines Namens, des Titels der Arbeit und der Name der Firma, bei der die Arbeit durchgeführt wurde - auf gedruckten Medien oder auf einer Internetseite (z.B. im Rahmen einer Auflistung der durch den Dozenten betreuten studentischen Abschlussarbeiten).
\\[2.5cm]

Kempten, den:  \hrulefill\enspace\enspace  Unterschrift:  \hrulefill

%\end{figure}
\end{document}

%\addcontentsline{toc}{section}{Anhang}
\newpage


\begin{document}

\newpage
\thispagestyle{empty}
\quad  \addtocounter{page}{-1}
\newpage

\end{document}
%\listoftables
%\addcontentsline{toc}{section}{Tabellenverzeichnis}
%\newpage


%\printbibliography[notkeyword=Quelle,title={\section*{Literatur}},heading=subbibliography]
%\addcontentsline{toc}{section}{Literaturverzeichnis}
%\newpage

%
\begin{document}
	

\section{Anhang}
%\addcontentsline{toc}{section}{Anhang}
%\markboth{\MakeUppercase {Anhang}}{}

% \section*{Abbildungsverzeichnis}

\listoffigures




%\addcontentsline{toc}{subsection}{Abbildungsverzeichnis}
%\listoffigures
%\newpage

% \listoftables
%\addcontentsline{toc}{subsection}{Tabellenverzeichnis}
%\listoftables
%\newpage	

%\begin{figure}



\subsection{Abbildungsverzeichnis}
%\makeatletter
%\@starttoc{lof}% Print List of Figures
%\makeatother
%\addcontentsline{toc}{subsection}{3.1 Abbildungsverzeichnis}
\listoffigures
\newpage

\subsection{Tabellenverzeichnis}
%\makeatletter
%\@starttoc{lot}% Print List of Tables
%\makeatother 
%\addcontentsline{toc}{subsection}{Tabellenverzeichnis}
\listoftables
\newpage

%\printbibliography[notkeyword=Quelle,title={\subsection*{Literaturverzeichnis}},heading=subbibliography]
\subsection{Literaturverzeichnis}
%\addcontentsline{toc}{subsection}{Literaturverzeichnis}
\printbibliography[heading=none]
%\clearpage


%\subsection*{Erklärung und Ermächtigung}
%%\section{Erklärung und Ermächtigung}
%\label{Erklärung und Ermächtigung}

%\begin{document}


\subsection{Erklärung}



Ich versichere, dass ich diese Abschlussarbeit selbstständig angefertigt, nicht anderweitig für Prüfungszwecke vorgelegt, alle benutzten Quellen und Hilfsmittel angegeben, sowie wörtliche und sinngemäße Zitate gekennzeichnet habe.
\\[2.5cm]

Kempten, den:  \hrulefill\enspace\enspace\enspace  Unterschrift:  \hrulefill
\\[2.5cm]

%\end{document}

\subsection{Ermächtigung}
\label{Ermächtigung}
%\begin{document}

Hiermit ermächtige ich die Hochschule Kempten und den betreuenden Dozenten, Herrn Prof. Dr. Jürgen Brauer, zur Veröffentlichung der Kurzzusammenfassung (Abstract) meiner Arbeit (im Wortlaut wie sie in der Arbeit enthalten ist) - einschließlich der Angabe meines Namens, des Titels der Arbeit und der Name der Firma, bei der die Arbeit durchgeführt wurde - auf gedruckten Medien oder auf einer Internetseite (z.B. im Rahmen einer Auflistung der durch den Dozenten betreuten studentischen Abschlussarbeiten).
\\[2.5cm]

Kempten, den:  \hrulefill\enspace\enspace  Unterschrift:  \hrulefill

%\end{document}
%\addcontentsline{toc}{subsection}{Erklärung und Ermächtigung}
\newpage
\subsection{Erklärung}



Ich versichere, dass ich diese Abschlussarbeit selbstständig angefertigt, nicht anderweitig für Prüfungszwecke vorgelegt, alle benutzten Quellen und Hilfsmittel angegeben, sowie wörtliche und sinngemäße Zitate gekennzeichnet habe.
\\[2.5cm]

Kempten, den:  \hrulefill\enspace\enspace\enspace  Unterschrift:  \hrulefill
\\[2.5cm]

%\end{document}

\subsection{Ermächtigung}
\label{Ermächtigung}
%\begin{document}

Hiermit ermächtige ich die Hochschule Kempten und den betreuenden Dozenten, Herrn Prof. Dr. Jürgen Brauer, zur Veröffentlichung der Kurzzusammenfassung (Abstract) meiner Arbeit (im Wortlaut wie sie in der Arbeit enthalten ist) - einschließlich der Angabe meines Namens, des Titels der Arbeit und der Name der Firma, bei der die Arbeit durchgeführt wurde - auf gedruckten Medien oder auf einer Internetseite (z.B. im Rahmen einer Auflistung der durch den Dozenten betreuten studentischen Abschlussarbeiten).
\\[2.5cm]

Kempten, den:  \hrulefill\enspace\enspace  Unterschrift:  \hrulefill

%\end{figure}
\end{document}

%\addcontentsline{toc}{section}{Anhang}
%\newpage

%%\section{Erklärung und Ermächtigung}
%\label{Erklärung und Ermächtigung}

%\begin{document}


\subsection{Erklärung}



Ich versichere, dass ich diese Abschlussarbeit selbstständig angefertigt, nicht anderweitig für Prüfungszwecke vorgelegt, alle benutzten Quellen und Hilfsmittel angegeben, sowie wörtliche und sinngemäße Zitate gekennzeichnet habe.
\\[2.5cm]

Kempten, den:  \hrulefill\enspace\enspace\enspace  Unterschrift:  \hrulefill
\\[2.5cm]

%\end{document}

\subsection{Ermächtigung}
\label{Ermächtigung}
%\begin{document}

Hiermit ermächtige ich die Hochschule Kempten und den betreuenden Dozenten, Herrn Prof. Dr. Jürgen Brauer, zur Veröffentlichung der Kurzzusammenfassung (Abstract) meiner Arbeit (im Wortlaut wie sie in der Arbeit enthalten ist) - einschließlich der Angabe meines Namens, des Titels der Arbeit und der Name der Firma, bei der die Arbeit durchgeführt wurde - auf gedruckten Medien oder auf einer Internetseite (z.B. im Rahmen einer Auflistung der durch den Dozenten betreuten studentischen Abschlussarbeiten).
\\[2.5cm]

Kempten, den:  \hrulefill\enspace\enspace  Unterschrift:  \hrulefill

%\end{document}
%\addcontentsline{toc}{section}{Erklärung und Ermächtigung}
%\newpage


%\thispagestyle{empty}
%\newpage
%\section*{}
%\newpage

\end{document}