\documentclass[12pt, bibliography=totoc, listof=totoc, listof=entryprefix]{scrartcl}

%Usepackages
\usepackage[german]{babel}
\usepackage[utf8]{inputenc}
\usepackage[T1]{fontenc}
\usepackage{color}
\usepackage{amssymb}
\usepackage{amsthm}
\usepackage{amsmath}
\usepackage{graphicx}
\usepackage{wrapfig}
%\usepackage[automark]{scrpage2}
\usepackage[automark]{scrlayer-scrpage}
\usepackage{color}
\usepackage{array}
\usepackage{standalone}
\usepackage{tabularx}
%\usepackage{tocloft}
\usepackage{url}
\usepackage{longtable}
\usepackage{hyperref}
\usepackage{chngcntr}
\usepackage{multirow}
\usepackage{enumitem}
%\usepackage{pdflscape}
\usepackage{nccmath}
\usepackage{acronym}
\usepackage{textcomp}
\usepackage{booktabs}
\usepackage{pdflscape}
\usepackage{listings}
\lstset{language=Pascal}
\renewcommand{\lstlistingname}{Code}
\usepackage{bytefield}
\usepackage[onehalfspacing]{setspace}
\usepackage{colortbl}
\usepackage{longtable}
\usepackage{subfigure}
\usepackage{tikz}
\usepackage{pgfplots}
\pgfplotsset{compat=1.10}
\usepackage{caption}
\usepackage{tabulary}
%Formale Gestaltung nach Studienamt Vorschlag
%Serifenschrift (Ähnelt Times New Roman)
\usepackage{times}
%\usepackage{helvet}
%Schriftgröße 12 s.o
%Zeilenabstand
\renewcommand{\baselinestretch}{1.30}
\usepackage[a4paper,lmargin={3.5cm},rmargin={1.5cm},tmargin={3.0cm},bmargin = {2.5cm}]{geometry}

%Bibliothek
\usepackage[backend=bibtex, style=numeric]{biblatex}
\usepackage[babel,german=quotes]{csquotes}

\addbibresource{Ramona_Literatur.bib}

%Grafiken
\usepackage[]{epstopdf}
\usepackage{float}
%\usepackage{gnuplottex}

%Kopfzeile
\pagestyle{scrheadings}
\clearscrheadfoot 
\ihead[\headmark]{\headmark}
\ofoot[\pagemark]{\pagemark}
\automark[section]{section}
\setheadsepline{0.4pt}

\headsep15mm

%Zeilenhöhe Tabelle
\renewcommand{\arraystretch}{1}

%nummerierung Abbildungen und Tabellen
\numberwithin{equation}{section}
\parindent0pt 
\renewcommand{\baselinestretch}{1.3}
\renewcommand{\thefigure}{\arabic{section}.\arabic{figure}}
\renewcommand{\thetable}{\thesection.\arabic{table}} 
\counterwithin{figure}{section}
\counterwithin{table}{section}


%Nummerierung Inhaltsverzeichnis
\setcounter{secnumdepth}{4}
\setcounter{tocdepth}{3}
\addtolength{\footskip}{-1cm}
%Für Formelverzeichnis
\DeclareNewTOC[%
 type=formel,
 name={Formel},%
 hang=5em,%
 listname={Formelverzeichnis}
]{for}
\newcommand*{\formelentry}[1]{%
 \addcontentsline{for}{formel}{\protect\numberline{\theequation} #1}%
}


\begin{document}
	

%\documentclass[standalone]

\thispagestyle{empty}
%\begin{document}


	\begin{titlepage}
	
			%\begin{tabular} {p{\textwidth}}
				
				
				
				\begin{flushright}	
					\includegraphics[scale=0.35]{img/hs.jpg}
			
					\Large{\textbf{{Hochschule Kempten\\}}}
					\footnotesize{University of Applied Sciences \\}
				\end{flushright}
				
				
				
				~\\
				
				\begin{center}
					\LARGE{MASTERTHESIS}
				\end{center}
				~\\
				\begin{center}
					\textbf{\large{Entwicklung und Evaluation neuartiger neuronaler Netzwerkarchitekturen für die Bildgenerierung im Bereich des unüberwachten Lernens\\}}
				\end{center}
				
				~\\
				
				\begin{center}
					vorgelegt von
				\end{center}
				
				\begin{center}
					\large{\textbf{Ramona Judith Neher}} \\
					\large{Matrikelnummer 313747}
				\end{center}
				
				
				\begin{center}
					am\\
					\large{01. März 2021}
				\end{center}
				
				\hspace{15cm}
				
				\begin{center}
					\begin{tabular}{lll}
						\textbf{Aufgabensteller:} & & \hspace{4cm} Prof. Dr.-Ing. Jürgen Brauer\\
						
						\textbf{Betreuer:} & & \hspace{4cm} Bonifaz Stuhr\\
						[10pt]
						\textbf{durchgeführt in der:} & & \hspace{4cm} Fakultät Informatik \\[10pt]
					\end{tabular}
				\end{center}
				
			%\end{tabular}

	\end{titlepage}
%\end{document}


\newpage

%Seitennummerierung Römisch
\pagenumbering{Roman}
\setcounter{page}{1}

\label{Sperrvermerk}

%\begin{document}

%\begin{bfseries}

%\[%keine Kopfzeile und Nummerierung
\section*{Sperrvermerk}
\markboth{\MakeUppercase {Sperrvermerk}}{}

Veröffentlichungen oder Vervielfältigungen der nachfolgenden Arbeit - auch nur auszugsweise oder in elektronischer Form - sind ohne ausdrückliche schriftliche Genehmigung nicht gestattet.
Hiervon ausgenommen ist eine Veröffentlichung der Kurzzusammenfassung (Abstract) der Arbeit (im Wortlaut wie sie in der Arbeit enthalten ist) - mit Angabe des Namens des Studierenden und des Titels der Arbeit - auf gedruckten Medien oder auf einer Internetseite (z.B. im Rahmen einer Auflistung der durch den Dozenten betreuten studentischen Abschlussarbeiten).

Die Sperrfrist gilt bis zum 02. Februar 2026.
Die Arbeit darf bis zum Ablauf der Sperrfrist nur für Prüfungszwecke verwendet werden.
 
%\]
%\end{bfseries}

%\end{document}
\addcontentsline{toc}{section}{Sperrvermerk}
\newpage

\section*{Abstract}

Die vorliegende Masterthesis befasst sich mit der \textit{Entwicklung und Evaluation neuartiger neuronaler Netzwerkarchitekturen für die Bildgenerierung im Bereich des unüberwachten Lernens}.

%Kurzbeschreibung, was in den einzelnen Abschnitten behandelt wird, ohne sachlich darauf einzugehen






\addcontentsline{toc}{section}{Abstract}
\newpage

\section*{Vorwort}

Diese Arbeit behandelt die \glqq Entwicklung und prototypische Umsetzung eines Konzeptes für ein ePlanungsboard zur verbrauchsgesteuerten Produktionsplanung\grqq. Sie wurde im Rahmen der Masterthesis meines Studiums an der Hochschule für angewandte Wissenschaften Kempten erstellt.

Die Idee zu dem Thema dieser Arbeit entstand in Zusammenarbeit mit Herr Stuhr. 

Welche Hardware habe ich genutzt, welchen Datensatz?

Ich durfte einige theoretische Grundlagen meines Studiums anwenden, die ich durch die Module. 

Ein besonderer Dank gilt meinem Betreuer Bonifaz Stur, der sich nicht nur immer Zeit genommen hat, mich zu unterstützen, sondern es als persönliches Anliegen sah, mich an mein Ziel zu führen. 
Auch meinem betreuenden Professor der Hochschule Kempten, Prof. Dr. Jürgen Brauer, möchte ich meinen Dank aussprechen. Auf eine zeitnahe und hilfreiche Antwort auf meine Fragen war immer Verlass. 
%Neben diesen beiden Personen bin ich auch meinem Chef, Dr. Arnd Kolleck, dankbar über das mir entgegengebrachte Vertrauen und die Möglichkeit zur Anfertigung meiner Arbeit in seinem Team.

Weiter möchte ich allen Personen danken, die mir bei Fragen und Problemen immer mit einer Antwort zur Seite standen und allen Korrekturlesern für ihren Aufwand und die Hilfestellung.


\addcontentsline{toc}{section}{Vorwort}
\newpage

\addcontentsline{toc}{section}{Inhaltsverzeichnis}
\tableofcontents
\newpage

\label{Abkürzungsverzeichnis}
\documentclass[20pt]{article}
\usepackage[utf8]{inputenc} % Direkte Eingabe von Umlauten und anderen Diakritika

\begin{document}

%\[%keine Kopfzeile und Nummerierung
\section*{Abkürzungsverzeichnis}
\markboth{\MakeUppercase {Abkürzungsverzeichnis}}{}

\begin{acronym} [MAXSHORT] %Auf wie viele Zeichen wird Aufgefüllt
	\acro {AdaGrad}{Adaptive Gradient}
	\acro {Adam}{Adaptive Moment Estimation}
	\acro {CNN}{Convolutional Neural Networks}
	\acro {D}{Discriminator}
	\acro {DCGAN}{Deep Convolutional Generative Adversarial Network}
	\acro {G}{Generator}
	\acro {GAN}{Generative Adversarial Network}
	\acro {KNN}{Künstliches Neuronales Netz}
	\acro {MLP}{Multilayer Perceptron}
	\acro {MSE}{Mean Squared Error}
	\acro {NN}{Neuronales Netz}
	\acro {ReLU}{Rectified Linear Unit}
	\acro {RMS}{Root Mean Square Propagation}
	\acro {tanh}{Tangens hyperbolicus}
\end{acronym}

%\]

\end{document}
\addcontentsline{toc}{section}{Abkürzungsverzeichnis}
\newpage

%Seitennummerierung arabisch
\pagenumbering{arabic}
\setcounter{page}{1}

\section{Einleitung}

\glqq die künstliche Intelligenz macht rasante Fortschritte und beeinflusst immer mehr Bereiche unseres Lebens.\grqq \cite{2}


\subsection{Problemstellung}
Ein bekanntes Verfahren der KI ist das Deep Learning. Deep Learning ist ein vielversprechender Ansatz. Es findet im Bereich des überwachten, sowie unüberwachten maschinellem Lernens große Anerkennung. Ein interessantes Modell in letzterem Bereich ist das Generativ Adversarial Network, kurz GAN. Dieses wurde 2014 von Ian Goodfellow vorgestellt. 
So stark (überzeugend) dieses Verfahren auch ist, sieht es sich dennoch einigen Problemen gegenübergestellt. 

\glqq beruhe der Entwurf neuronaler Netze auf einer großen Willkür: Bei der Entscheidung, wie viele Lagen mit wie vielen Neuronen genutzt werden sollten, beruhe vieles auf Bauchgefühl oder auf Ausprobieren.\grqq{} \cite[S.44]{3}

\subsection{Zielsetzung}
Diese Arbeit wird neuartige Netzwerkarchitekturen für die Bildgenerierung vorstellen. Basis bilden hierbei bekannte Architekturen. Diese werden ausgewertet ob und wie weit Generative Adversarial Networks dadurch effizienter trainiert werden können. Das Ziel erfolgreichen Trainings ist die Generierung täuschend echter Bilder.  
\newpage

\section{Related Work}

Was haben andere bisher publiziert? Worauf beziehe ich mich
\newpage

\section{Künstliche Neuronale Netze}
\label{KNN}
\begin{document}
Die \textit{künstlichen neuronale Netze} oder auch \textit{künstlichen neuronale Netzwerke} (KNN) sind Methoden im Bereich des Deep Learnings, welche die Realität repräsentieren und daher als Modelle bezeichnet werden \cite[vlg.][S. 40]{13}. Mit diesen Modellen konnten bereits beachtliche Erfolge erzielt werden. Aus enormen Datenmengen extrahieren die Netze \glqq Regelmäßigkeiten, Muster oder Modelle\grqq{} \cite[S. 40]{13}. 
Sie generieren also \glqq Wissen aus Erfahrung\grqq{} \cite[S. 40]{13} und \glqq der Computer [ist so in der Lage,] eigene Schlussfolgerungen aus diesem Wissen [zu] ziehen\grqq{} \cite[S. 33]{12}.
Hierdurch entdecken die \textit{KNN} nicht nur schneller Problemlösungen, sondern auch solche, die für den Menschen nicht zu erkennen sind. \cite[vgl.][S. 40]{13}

Ein \textit{künstliches neuronales Netz} setzt sich aus verschiedenen Schichten (engl. \textit{layers}) zusammen:
fungiert!!!!!!!!!!!!!!!!!!!!!!!!!!!!!!!!!!!!
\begin{itemize}
	\item Zu Beginn des Netzes findet sich die \textit{Eingabeschicht} (engl. input layer). Diese repräsentiert die Merkmale, auch Feature genannt, des zu verarbeitenden Inputs als numerischen Wert. Es wird auch von einem $n$-dimensionalen Input- oder Merkmalsvektor $X$ mit den Werten $x\textsubscript{n}$ gesprochen \cite[vlg.][S.176]{14}. Im Falle eines Bildes beispielsweise entspricht ein Pixel einem Merkmal.
	\item Darauf folgen die \textit{verdeckten Schichten} (engl. hidden layer). Sie dienen der Weiterverarbeitung der Daten.
	\item den Abschluss bildet die \textit{Ausgabeschicht} (engl. output layer). Sie stellt die Zielwerte der Aufgabe dar. \\
	\cite[vgl. ][S.72]{12}
\end{itemize}
Die Schichten wiederum bestehen aus künstlichen Neuronen, die mittels einer gewichteten Verbindung verknüpft sind. Gewichtet bedeutet, dass eine Verbindung selbst einen Wert besitzt. Dieser Wert regelt dann \glqq den Anteil des Eingangswertes auf die Eingangssumme.\grqq{} \cite[S. 28]{13}, damit hängt die Prädiktion eines Netzes direkt von den Gewichtungen, auch Modellparameter $\Theta$ genannt, ab \cite[vgl.][S. 77]{12}. Die Gewichtungen einer Schicht mit $m$ Neuronen können auch als ein Vektor $w \in \mathbb{R}^m$ abgebildet. Somit ist es möglich einen Input der nachfolgenden Schichten durch Vektormultiplikation zu kalkulieren.

Nicht nur in einem Neuron, auch im gesamten Netz finden einige Vorgänge statt, die dazu beitragen, dass ein KNN so mächtig ist. Diese sollen nachfolgend näher gebracht werden.

\subsection{Das künstliche Neuron}
\label{Neuron}
Wie bereits erwähnt, befinden sich künstliche Neuronen in den einzelnen Schichten eines \textit{künstlichen neuronalen Netzes}. Deren Anzahl variiert hierbei von Schicht zu Schicht, jedoch ist jedes Neuron einer Schicht mit allen Neuronen der folgenden Schicht gewichtet verbunden. Dabei ist der Input eines Neurons die Summe der gewichteten Outputs der vorgelagerten Neuronen \cite[vgl.][]{17}. Die Summe wird anschließend auf eine sogenannte \textit{Aktivierungsfunktion} {siehe Kapitel \ref{Aktivierungsfunktion}} angewendet. Damit stellt ein Neuron eine kleine \textit{Berechnungseinheit} dar \cite[vgl.][]{13} die durch folgende Formel definiert ist:

\begin{equation}
y = f(\sum_{i=0}^{n} w\textsubscript{i}*x\textsubscript{i})
\end{equation} 

welche in Abbildung \ref{fig:Perzeptron} veranschaulicht dargestellt werden soll. x\textsubscript{n} beschreibt dabei einen Input, w\textsubscript{n} die Gewichtung und o\textsubscript{1} einen Output. 

\begin{figure}[H]
	\centering
	\includegraphics[width=0.3\textwidth]{img/Perzeptron.png}
	\caption{Perzeptron, eigene Darstellung}
	\label{fig:Perzeptron}
\end{figure}

Eine Ausnahme hiervon bildet die Eingabeschicht, denn wie bereits erwähnt, handelt es sich bei diesen Neuronen um den Merkmalsvektor, bei welchem ein Neuron einem Merkmal entspricht \cite[vgl.][S.177]{14} 
Die Eingabewerte $x\textsubscript{n}$ bilden also unverändert die erste Schicht \cite[vgl.][S.178]{14} . Nachfolgende Graphik zeigt ein extrem vereinfachtes KNN, anhand dessen die in den einzelnen Neuronen stattfindenden Vorgänge visualisiert:

\begin{figure}[H]
	\centering
	\includegraphics[width=0.6\textwidth]{img/NN-Berechnung.png}
	\caption{Ein einfaches KNN, eigene Darstellung}
	\label{fig:NN-Berechnung}
\end{figure}

Abbildung \ref{fig:NN-Berechnung} zeigt, dass die Neuronen der Eingabeschicht lediglich einen Inputvektor der Größe $\mathbb{R}^3$ darstellen. Die mathematische Manipulation der Daten beginnt in der folgenden Schicht, der ersten verdeckten, und endet mit der Ausgabeschicht. In dieser befinden sich zwei Neuronen. Das bedeutet, dass \glqq aus einer dreidimensionalen Eingabe zwei Größen [prädiziert]\grqq{} \cite[S.174]{14} wurden. Der Wert der Prädiktion soll in kommendem Kapitel näher diskutiert werden.

\subsection{Aktivierungsfunktion} 
\label{Aktivierungsfunktion}
Ein Neuron leitet ein Signal nur dann weiter, wenn dieses einen gewissen Schwellenwert übersteigt. Dafür bildet die Summe des Eingabevektors mit den je zugehörigen Gewichtungen den Parameter der \textit{Aktivierungsfunktion}. Zusätzlich normalisiert diese die Werte, die durch das Netz fließen. So wird verhindert, dass ein drastisch unterschiedlicher Wertebereich entsteht \cite[vlg.][]{18}. 
Wird der Schwellenwert übertroffen, wird das Neuron aktiviert und das Ergebnis bildet den zu übertragenden Ausgang. Je nach Funktion können unterschiedliche Resultate erwartet werden. Die Bedeutendsten im Bereich der \textit{künstlichen neuronalen Netze} sind nachstehend mit zugehörigem Wertebereich aufgelistet. \cite[vgl.][S. 70]{12} \cite[vgl.][S. 35]{13} 

\begin{table}[h]
	\begin{tabular}[h]{p{1cm}|p{4cm}|p{3.9cm}|p{5.1cm}}
		Index & Bezeichnung & Formel & Wertebereich der Ausgabe \\
		\hline
		\rule{0pt}{2em}
		1 & Sigmoid & ${f(x)= \frac {1}{1+e\textsuperscript{-x}}}$ & [0, 1] \\
		\rule{0pt}{2em}
		2 & Rectified Linear Unit (ReLU) & ${f(x) = max(0,x)}$ & 
		$f(x)= \begin{cases}
		0, \mbox{ für } x < 0 \\ x, \mbox{ für } x \ge 0 
		\end{cases}$ \\
		\rule{0pt}{2em}
		3 & LeakyReLU & ${f(x) = max(\alpha x,x)}$, mit $\alpha$ = 0.01 &
		$f(x)= \begin{cases}
		0.01x, \mbox{ für } x < 0 \\ x, \mbox{ für } x \ge 0 
		\end{cases}$ \\
		\rule{0pt}{2em}
		4 & Tangens hyperbolicus (tanh) & ${f(x)= \frac {e\textsuperscript{x} - e \textsuperscript{-x}} {e\textsuperscript{x} + e\textsuperscript{-x}}}$ & [-1, 1] \\
		\rule{0pt}{2em}
		5 & Softmax & $f(x)= \frac{e\textsuperscript{xi}}{\sum_{j=1}^{N} e\textsuperscript{xj}}$ mit $i = 1,...,N$ & [0, 1]
	\end{tabular}
	\captionsetup{justification=centering}
	\caption{Aktivierungsfunktion mit zugehöriger Formel und Wertebereich \\ Daten in Anlehnung an \cite[vgl.][]{18}}
	\label{tab:Aktivierungsfunktion}
\end{table}

Die Auswahl der passenden Funktion hängt nicht nur von ihrem Wertebereich ab. Zu aktuellem Stand der Forschung ist nicht immer gegeben, dass eine spezielle Methode eine hinreichende Lösung für gleiche Probleme ist. Daher ist es ratsam, ein \textit{künstliches neuronales Netz} mit unterschiedlichen Methoden zu testen, trainieren und anschließend zu evaluieren.
\\
Daneben sollte die Wahl einer Funktion der Ausgabeschicht genauer betrachtet werden. Hier ist ein essenzieller Punkt, der zur Entscheidung beiträgt, das Klassifikationsproblem. %Die Neuronen bilden hier die Zielwerte der Aufgabe ab. 
In den meisten Netzen finden sich die \textit{Sigmoidfunktion}, die \textit{tanh-Funktion} oder die \textit{Softmaxfunktion} in dem Output-Layer.

\begin{itemize}
	\item \textit{Sigmoid}: findet Verwendung in der Wahrscheinlichkeitsschätzung \cite[vgl.][S. 18]{7}. Der Output kann so interpretiert werden, wie sicher das Modell ist, dass ein spezieller Input einer Klasse angehört oder nicht.
	\cite[vlg.][]{19}
	
	\item \textit{Softmax}: Die Werte einer Schicht werden hier nicht nur auf einen Wertebereich [0,1] normalisiert, sondern ordnet jede Ausgabe so zu, dass die Gesamtsumme 1 ergibt. Dies gleicht einer Wahrscheinlichkeitsverteilung. \cite[vgl.][]{18} 	
\end{itemize}

\subsection{Lernprozess}
\label{Lernprozess}
Das Ziel eines Modells ist es, eine hinreichend präzise Abbildung des Inputs auf einen gegebenen Output zu prädizieren, also Eingabedaten realitätsnah zu klassifizieren. Dies erreicht ein Modell dann, wenn alle Gewichtungen des Netzes ihrem optimalen Wert entsprechen. 

Zu Beginn werden die Gewichtungen üblicherweise mit zufälligen Werten initialisiert \cite[vgl.][S. 166]{13} \cite[vgl.][]{20}, welche im Laufe des Lernprozesses sukzessiv angepasst werden.

Um dies zu erreichen wird zunächst mittels einer \textit{Fehlerfunktion} ein Fehler an der Ausgabeschicht \cite[vgl.][]{20} ermittelt. Anschließend muss dieser rückwärts gerichtet durch das Netz geführt werden, um so die einzelnen Gewichtungen durch ein Gradientenabstieg der \textit{Fehlerfunktion} anzupassen. 

Diese mathematischen Vorgehen kombiniert, werden als \textit{Backprogagation} bezeichnet und nachstehend näher erläutert. \cite[vgl.][]{20}

\subsubsection{Fehlerfunktion} 
\label{Fehlerfunktion}
Eine \textit{Fehlerfunktion} (engl. \textit{loss-function}), auch \textit{Kostenfunktion} (engl. \textit{cost-function}) genannt, bestimmt also die \glqq Diskrepanz zwischen errechnetem und erwartetem Output\grqq{} \cite[S. 77]{12} in der Ausgabeschicht. Formel \ref{Eq:ff} definiert die einfachste Form der Funktion:

\begin{equation} \label{Eq:ff}
	E = y - \hat{y}
\end{equation}
mit $E$ als Fehlerwert, $y$ als Soll- und $\hat{y}$ als Ist-Wert \cite[vgl.][S. 161]{13}.
\\ 

Neben des Ausdrucks \ref{Eq:ff} bestehen zahlreiche weitere \textit{Fehlerfunktionen}, wobei die Komplexität meist zunimmt und die Aufgabengebiete variieren. Nachstehend werden einige bekannte Funktionen kurz vorgestellt:\\

Die \textbf{Mean Squared Error - Funktion} (MSE) ermittelt den mittleren quadratischen Fehler. Sie eignet sich beispielsweise für Regressionsprobleme:
\begin{equation} 
	E = \frac{1}{n} \sum_{i=1}^{n} (y\textsubscript{i} - \hat{y}\textsubscript{i})^{2}
\end{equation}
\cite[vgl.][S. 76 f.]{12} \\

Die \textbf{Categorical Crossentropy} findet Verwendung in Klassifikationsaufgaben, bei welchen ein Input einer von mehreren möglichen Klassen zugeordnet werden soll. Diese \textit{Fehlerfunktion} ermittelt einen \textit{One-Hot-Vector} als Output, welcher mit dem realen Vektor abgeglichen wird. \cite[vgl.][]{10, 12}
\begin{equation} 
	E = - \sum_{i=1}^{n} y\textsubscript{i} * log(\hat{y}\textsubscript{i}) 
\end{equation}
\cite[vgl.][]{23} \\


Auch die \textbf{Binary Crossentropy} wird auf Klassifikationsprobleme angewendet, jedoch handelt es sich hier um eine Zuordnung zu lediglich zwei möglichen Klassen. Die Ausgabeschicht besteht aus einem Neuron, welches eines der zwei Label abbildet. Dadurch entspricht die Vorhersage nur einem Skalar, welcher aussagt, wie sicher das Modell einen konkreten Input dieser Klasse zuordnet. 
Strebt die Prädiktion des Modells gegen 1, so ordnet es die Eingabedaten der Klasse zu, wohingegen ein streben gegen 0 bedeutet, dass die Daten der anderen Klasse zugewiesen werden. 
\begin{equation} 
	E = -\frac{1}{n} \sum_{i=1}^{n} y\textsubscript{i} * log(\hat{y}\textsubscript{i}) + (1-y\textsubscript{i}) * log(1-\hat{y}\textsubscript{i})
\end{equation}
\cite[vgl.][]{22} \\

Die Werte $E$, $y$ und $\hat{y}$ der folgenden Gleichungen wurden bereits vorgestellt. Hinzu kommt die Variable $n$. Diese entspricht der Dimension des Outputvektors $o = \mathbb{R}^n$
\\

Das Ziel ist es, die Differenz $E$ zu minimieren, was durch die kontinuierliche Anpassung der Gewichtungen erreicht wird. Das bedeutet, dass die \textit{Zielfunktion} \glqq in Bezug auf die Parameter\grqq{} \cite[S. 185]{14} optimiert wird \cite[vgl.][]{21} \cite[vgl.][S. 76]{12}
\glqq Angefangen mit den Gewichtungen in der letzten Verbindung zwischen dem Output-Layer und dem vorherigen Layer, werden die Anpassungen Layer für Layer in Richtung der Input-Layer vorgenommen.\grqq{} \cite[S. 78]{12}

Diese sukzessive Angleichung wird durch das sogenannte \textit{Gradientenverfahren} erreicht.

\subsubsection{Gradientenverfahren}
\glqq Gradient descent [zu deutsch: Gradientenverfahren] is one of the most popular algorithms to perform optimization and by far the most common way to optimize neural networks.\grqq{} \cite{21}

Das \textit{Gradientenverfahren} (engl. gradient descent) ist ein mathematisches Verfahren, um Optimierungsprobleme zu lösen. Ein \textit{Gradient} bestimmt die Richtung der größten Änderung. Da die Optimierung einer \textit{Fehlerfunktion} darauf beruht, diese zu minimieren, wird das \textit{Gradientenverfahren} angewandt, um das globale Minimum zu finden. Es ist hier also die größte negative Änderung von Bedeutung. Dieser Wert wird durch die Ableitung einer Funktion an der Stelle $x$ ermittelt. Folgende allgemeine Gleichung definiert die Anpassung der Gewichte: (Formel von Buch ID 12, S.77. Hinweis auf Batchgröße miteinbringen?)
\begin{equation} 
	w\textsubscript{i\textsubscript{new}} = w\textsubscript{i\textsubscript{old}} - \eta * \frac{\partial E}{\partial w\textsubscript{i}}
\end{equation}
\glqq Die neue Gewichtung ergibt sich aus der Differenz zwischen der Gewichtung, die angepasst werden soll, und der partiellen Ableitung des berechneten Fehlers $E$ in Bezug auf die Gewichtungen $w\textsubscript{i}$\grqq{} \cite[S. 77]{12}. Die Ableitung wird mit dem Parameter $\eta$ multipliziert. Dieser wird als \textit{Lernparameter} bezeichnet und bestimmt die Schrittweite, mit welcher die Werte der Modelparameter sich dem Minimum annähern.
\cite[vgl.][]{21} \\

Wurde dies erreicht, kann davon ausgegangen werden, dass die Parameter ihren bestmöglichen Wert erreicht haben. \cite[vgl.][]{12, 13, 21}. 

Abbildung \ref{fig:Gradientenverfahren} stellt die Funktion eines Gewichtes $w\textsubscript{i} \in \Theta = (w\textsubscript{1}, ..., w\textsubscript{n})$ dar und veranschaulicht damit das Prinzip des \textit{Gradientenverfahrens}. 

\begin{figure}[H]
	\centering
	\includegraphics[width=0.6\textwidth]{img/Graph-Gewicht.png}
	\caption{Gradientenverfahren, eigene Darstellung}
	\label{fig:Gradientenverfahren}
\end{figure}

Der Startpunkt entspricht dem zufälligen Startwert der Gewichtung. Dieser Punkt soll im Laufe des Trainings sukzessive dem globalen Minimum annähern.

Tatsächlich ist dieses Verfahren um ein vielfaches komplexer, da es sich nicht wie in Abbildung \ref{fig:Gradientenverfahren} um ein zweidimensionalen Raum $\mathbb{R}^2$ handelt, sondern um einen multidimensionalen Raum $\mathbb{R}^n$. \cite[vgl.][S. 79]{12}

\subsubsection{Herausforderungen des Lernens} \label{HerausforderungenDesLernens}
Aufgrund der Komplexität ist das Training eines Netzes vielen Herausforderungen gegenüber gestellt: \\

Das Finden des \textbf{globalen Minimums} stellt eine signifikante Problematik dar. Es besteht hier das Risiko, dass sich die Parameter in einem lokalen Minimum einpendeln, oder einen Sattelpunkt erreichen. In beiden Fällen stagniert die Anpassung und der Idealwert kann nicht erreicht werden. \cite[vgl.][]{21} 
Dieses Problem ist kaum zu umgehen, jedoch kann mithilfe unterschiedlicher Startwerte unterschiedliche Minima gefunden werden. Damit wird die Wahrscheinlichkeit erhöht, das globale Minimum zu finden.\\

Das \textbf{Overfitting} ist ein Phänomen, welches eintritt, wenn sich ein Modell zu sehr an die Trainingsdaten angepasst hat und aufgrund dessen keine Generalisierung stattfindet. Es wird davon gesprochen, dass das Modell die Trainingsdaten \glqq auswendig gelernt\grqq{} hat \cite[vgl.][S.82]{12}. Um diesem Problem entgegenwirken, können zwei interessante Methoden angewendet werden. Zum einen das \textit{Early Stopping}, welches das Training an einem Punkt beendet, obwohl der Fehler weiter reduziert werden könnte, jedoch ab diesem Punkt wieder zunehmen würde. Und zum anderen das \textit{Dropout}. Bei diesem Verfahren werden in der Eingabe- sowie den verdeckten Schichten eine gewisse Anzahl an zufällig ausgewählten Neuronen nicht mitberechnet. Abbildung \ref{fig:Dropout} veranschaulicht dieses Prinzip \cite[vgl.][S. 216 f.]{13} 

\begin{figure}[H]
	\centering
	\includegraphics[width=0.5\textwidth]{img/Dropout.png}
	\caption{Prinzip des Dropouts, eigene Darstellung}
	\label{fig:Dropout}
\end{figure} 


Von \textbf{Vanishing Gradients} wird gesprochen, wenn die \textit{Fehlergradienten} mit Annäherung an die Eingabeschicht immer kleiner werden. Dadurch wird eine Lösung nicht oder nur sehr langsam erzielt \cite[vgl.][S. 210]{13}. Es empfiehlt sich, eine \textit{Aktivierungsfunktion} zu wählen, die diesem Problem direkt entgegenwirken kann, wie beispielsweise die \textit{Leaky ReLU}. \cite[vgl.][S. 212]{13} \\

Auch die \textbf{Wahl der Hyperparameter} ist eine bedeutende Thematik. \textit{Hyperparameter} beschreiben dabei entscheidende Parameter, die ein Netz definieren \cite[vgl.][S. 202]{13}.
Jedoch ist die ideale Abstimmung der \textit{Hyperparameter} aktuell Gegenstand der Forschung. Bislang kann keine Antwort gegeben werden, welche Einstellung der Parameter eine hinreichende Lösung für bestimmte Problemstellungen ausmacht \cite[vgl.][]{24}. Daher ist es sinnvoll einige Testläufe durchzuführen und anschließend zu evaluieren, um die \textit{Hyperparameter} für einen konkreten Fall ideal festlegen zu können. \\	

Um eine Konvergenz des Lernprozesses der \textit{künstlichen neuronalen Netze} zu begünstigen, werden sogenannte \textit{Optimierer} (engl. Optimizer) angewendet. \glqq Diese Algorithmen unterscheiden sich im Wesentlichen in der Art und Weise, wie sie mit den verfügbaren Parametern zum Lernverfahren umgehen\grqq {} \cite[S. 80]{12}.

Algorithmen, die sich je nach Art und Weise des Netzes als geeignet erwiesen, sind unter anderem:

\begin{itemize}
	\item Momentum Optimierer
	\item \textbf{Ada}ptive \textbf{Gra}dient (AdaGrad)
	\item \textbf{R}oot \textbf{M}ean \textbf{S}quare Propagation\textbf{Prop}agation (RMSprop) 
	\item \textbf{Ada}ptive \textbf{M}oment Estimation (Adam)
\end{itemize}
\cite[vgl.][S. 80]{12}


\subsubsection{Batchgröße}
Ein weiterer essenzieller \textit{Hyperparameter}, der die Dynamik eines Netzes beeinflusst, bildet die \textit{Batchgröße}. Dieser Parameter bestimmt die Anzahl von Exemplaren der Trainingsdaten, die eine Teilmenge darstellen. Diese wird auch \textit{Batch} genannt. Es handelt sich dabei um eine konstante Größe, während die Beispieldaten meist zufällig aus dem gesamten Satz gruppiert werden. Dieser \textit{Hyperparameter} beeinflusst nicht nur die Lerngeschwindigkeit eines Modells, sondern auch die Stabilität des Lernprozesses.

Wurden die Gewichtungen des Netzes mittels einer Untermenge adaptiert, folgt die nächste Anpassung mit einer differenten Teilmenge. Dies wiederholt sich, bis so der gesamte Trainingssatz einmal durchlaufen wurde. \glqq Den Zyklus, bei dem der Algorithmus einmal durch all diese Trainingsdaten durchgelaufen ist, nennt man \textit{Epoche}.\grqq {} \cite[S. 75]{12}

Dabei sind drei Begriffe zu unterscheiden:
\begin{itemize}
	\item \textbf{Batch Gradient Descent:} Batchgröße = gesamter Trainingsdatensatz
	\item \textbf{Stochastic Gradient Descent:} Batchgröße = 1 
	\item \textbf{Mini-Batch:} 1 < Batchgröße < gesamter Trainingsdatensatz
\end{itemize}

Nicht immer führt eine enorme Datenmenge zu optimalem Ergebnis. Dies spiegelt sich auch bei der \textit{Batchgröße}. Kleinere \textit{Batches} resultieren durch ihr rauschen in einer effektiveren Generalisierung und verfügen über einen deutlicheren Regularisierungseffekt.
Zudem sollte die Größe den Speicheranforderungen der Hardware entsprechen. Daher wird weitgehend von einer Anzahl von 32, 64 oder 128 Exemplaren pro \textit{Batch} gesprochen.

\cite[vgl.][]{25, 26}

\subsection{Arten künstlicher neuronaler Netze}
Mittlerweile besteht eine enorme Diversität von \textit{künstlichen neuronalen Netzen}, welche sich jedoch alle in ähnlicher Weise den bisher definierten Prinzipien bedienen.
Nachstehend sind in Kürze zwei wesentliche Varianten vorgestellt.

\subsubsection{Multilayer Perzeptron}
Ein \textit{Perzeptron} setzt sich in seiner einfachsten Variante aus einem Neuron mit dessen gewichtetem Input, einer Aktivierungsfunktion und dem daraus resultierendem Output zusammen. Es wurde 1957 von Frank Rosenblatt entwickelt um im Jahre 1958 vorgestellt. Es gleicht der unter Abbildung \ref{fig:Perzeptron} präsentierten Struktur.

Von einem \textit{Multilayer Perzeptron} (MLP) wird dann gesprochen, wenn es sich um ein mehrschichtiges Netz handelt. Informationen werden vorwärts gerichtet von der Eingabeschicht über die verdeckten Schichten zur Ausgabeschicht geleitet. 

Die einzelnen Schichten des MLPs sind dabei untereinander vernetzt, wobei Neuronen derselben Schicht keine Verbindung vorweisen.
\cite[vgl.][]{15, 17}  \cite[vgl.][S. 80, S. 144]{13}
\\
Der Aufbau eines MLPs ähnelt also der unter Kapitel \ref{KNN} und \ref{Neuron} beschriebenen Architektur. Diese wird durch Abbildung \ref{fig:Netz} schematisch dargestellt, wobei \glqq x\textsubscript{x}\grqq{} die Eingabeneuronen, \glqq h\textsubscript{xx}\grqq{} die verdeckten Neuronen und \glqq o\textsubscript{1}\grqq{} das Ausgangsneuron repräsentiert.

\begin{figure}[H]
	\centering
	\includegraphics[width=0.6\textwidth]{img/NN.png}
	\caption{Architektur eines multilayer Perzeptrons, eigene Darstellung}
	\label{fig:Netz}
\end{figure}

%\begin{figure}[H]
%	\centering
%	\includegraphics[width=0.6\textwidth]{img/NN-gelernt.png}
%	\caption{Verbindungen eines trainierten multilayer Perzeptrons, eigene Darstellung}
%	\label{fig:NN-gelernt}
%\end{figure}

\subsubsection{Convolutional Neural Network}
Das \textit{Convolutional neural Network} (CNN) ist eine bedeutende Art des \textit{künstlichen neuronalen Netzwerks}. Sie \glqq sind in der Bilderkennung die State-of-the-Art-Methode\grqq{} \cite{29} und dies mit guten Gründen:

\begin{itemize}
	\item Sie verarbeiten selbst enorme Mengen an Eingabedaten äußerst erfolgreich
	\item CNNs sind gegenüber Verzerrungen oder anderen optischen Veränderungen unempfindlich
	\item Auch bei Bildern mit verschiedenen Lichtverhältnissen und unterschiedlichen Perspektiven können typische Merkmale problemlos extrahiert werden.
	\item Sie weisen einen wesentlich geringeren Speicherplatzbedarf auf
	\item Dank neuester Hardware kann ein effizienter Trainingsprozess sichergestellt werden
\end{itemize}

Klassische KNN, wie das MLP, sind starke Verfahren. Würden diese Modelle jedoch für die Verarbeitung von extrem großen Inputdaten verwendet werden, wie beispielsweise für die Bildverarbeitung, würde die Summe an Neuronen und Gewichtungen enorme Ausmaße annehmen.
Das CNN bedient sich eines Prinzips, durch welches sich die Komplexität erheblich reduzieren lässt. Neuronen dieser Netze teilen sich Verbindungen, \glqq wobei jede nachfolgende Schicht nur auf einen lokalen Bereich der Vorgängerschicht reagiert\grqq \cite[S. 198]{13}. 

Dies ermöglicht einen effizienten Trainingsprozess, trotz extremen Eingabedaten. \\

Die Struktur dieser Modelle besteht aus zwei Teilen: an erster Stelle der Kodierungs- und folgend der Prädiktionsblock.

In Ersterem findet die sogenannte \textit{Faltung} (engl. convolution) statt. Das Eingabebild wird als Matrix dargestellt, ebenso wie die Gewichtungen, welche als \textit{Filter} fungieren und typischerweise einer 3x3- oder 5x5-Matrix entsprechen. Jeder \textit{Filter} extrahiert dabei ein bestimmtes Merkmale, wie Linien, Kanten, Formen etc., des zugrundeliegenden Inputs.

Dieser ist zu Beginn das Eingangsbild. Wurden die Gewichtungen auf dieses angewendet, entsteht eine neue Ebene, auch \textit{Feature Map} genannt. Diese dient als Input für die folgende Faltung, wobei während einer Faltung mehrere \textit{Filter} angewendet werden können. Eine Ebene stellt dabei ein ausgelesenes \textit{Merkmal} dar.
Die \textit{Feature Maps} einer Faltung Gruppiert werden als \textit{Convolutional Layer} bezeichnet.

Die Gewichtungen bewegen sich in einer festgelegten Schrittweite über den Input. Bei jedem Schritt werden die überlappenden Werte multipliziert und die Produkte summiert. Somit werden die \textit{Feature} sukzessive extrahiert und eine neue Ebene entsteht. Abbildung \ref{fig:multiplikation} stellt die Berechnung exemplarisch dar: 
\cite[vgl.][]{27}
\begin{figure}[H]
	\centering
	\includegraphics[width=0.4\textwidth]{img/CNN-multiplikation.png}
	\caption{Berechnung CNN, eigene Darstellung in Anlehnung an \cite{28}}
	\label{fig:multiplikation}
\end{figure}

daraus entsteht folgende Formel:

\begin{equation} 
x\textsubscript{11} * w\textsubscript{1} +
x\textsubscript{12} * w\textsubscript{2} +
x\textsubscript{21} * w\textsubscript{3} +
x\textsubscript{22} * w\textsubscript{4} + ... =
\end{equation}

Abbildung \ref{fig:Faltung} veranschaulicht das Prinzip der Faltung eines Farbbild:

\begin{figure}[H]
	\centering
	\includegraphics[width=1\textwidth]{img/Faltung.png}
	\caption{Berechnung CNN, eigene Darstellung in Anlehnung an \cite{28}}
	\label{fig:Faltung}
\end{figure}

Einer Faltung folgt für gewöhnlich ein \textit{Pooling}, welches ebenso eine Reduktion der Dimensionen des Inputs bewirkt. Meist wird hierfür das \textit{Max Pooling} verwendet. Dabei wird innerhalb eins 2x2- oder 3x3-Filter, der maximale Wert bestimmt. Diese Werte bilden den Input des folgenden Vorgangs. \cite[vgl.][S. 204]{13}

In Abbildung \ref{fig:Pooling} wird ein 2x2-Filter auf einen 4x4-Input angewendet. Die überflüssigen Informationen eines Filters werden verworfen, sodass nur der höchste Wert weitergeleitet wird.

\begin{figure}[H]
	\centering
	\includegraphics[width=0.4\textwidth]{img/Pooling.png}
	\caption{Max Pooling, eigene Darstellung in Anlehnung an \cite{28}}
	\label{fig:Pooling}
\end{figure}

Der Vorteil des Poolings besteht in der Reduktion der Datenmenge. Dies gewährleistet eine gesteigerte Berechnungsgeschwindigkeit trotz Erhalt der Leistungsfähigkeit. \cite[vgl.]{29}

Der Prozess von Faltung, gefolgt von Pooling, kann unbegrenzt häufig durchgeführt werden. Solange, bis das verfolgte Ziel, einen Input ausreichend beschreiben zu können, erreicht ist.
\cite[vgl.][]{27}

Daraufhin folgt der Prädiktionsblock. Hier wird das sogenannte \textit{Flatten} auf den letzten \textit{Convolutional Layer} angewendet, also in einen Vektor kodiert.
Die Abbildung \ref{fig:Flatten} veranschaulicht das Verfahren des \textit{Flattens}.

\begin{figure}[H]
	\centering
	\includegraphics[width=0.6\textwidth]{img/Flatten.png}
	\caption{Flatten einer 3x3 Matrix, eigene Darstellung}
	\label{fig:Flatten}
\end{figure}

Anschließend dient der Vektor als Input eines üblichen KNN \glqq das die kodierte Repräsentation des Bildes klassifiziert\grqq. \cite[S. 199]{13}

Nachstehend ist die gesamte Struktur eines \textit{convolutional neural Networks} anschaulich dargestellt:

\begin{figure}[H]
	\centering
	\includegraphics[width=1\textwidth]{img/CNN-Struktur2.png}
	\caption{Struktur eines CNN, eigene Darstellung in Anlehnung an \cite{13, 28}}
	\label{fig:CNN-Struktur}
\end{figure}

\cite[vgl.][]{13,27,28,29}
\end{document}
\newpage

\input{GAN Funktionsweise}
\newpage

\section{GAN - Generative Adversarial Network} \label{GAN}
\begin{document}

\glq \glqq the coolest idea in deep learning in the last 20 years\grqq{} \grq{} \cite{0}
So beschrieb Yann LeCun, Leiter der Facebook KI-Abteilung in New York, das von Ian J. Goodfellow in 2014 vorgestellte Verfahren gegenüber der Carnegie Mellon University in Pittsburgh, Pennsylvania. \cite[vgl.][]{0} \\


Es handelt sich um ein neuartiges \textit{deep neural network} im Bereich des unüberwachten Lernens - das \textit{Generative Adversarial Network}. Dabei stehen sich zwei konkurrierende Modelle gegenüber:
Auf der einen Seite der \textit{Generator} (engl. generator), ein \textit{generative} Modell, welches gefälschte Daten erstellt. Auf der anderen Seite der \textit{Diskriminator} (engl. discriminator), ein \textit{diskriminatives} Modell. Dessen Sinn ist es, die generierten Daten von realen Beispielen unterscheiden zu können und dient demnach als Klassifikator.
Während sich die Modelle gegenseitig trainieren, verfolgt das \textit{generative} Modell das Ziel, das \textit{diskriminative} Modell zu täuschen, sodass dieses die generierte Daten nicht von echten Beispielen differenzieren kann.
Die Zielsetzung eines GANs ist es, täuschend echte Daten maschinell zu erstellen.
Ian J. Goodfellow präsentierte die Netzwerke im Paper \cite{4} als \textit{multilayer Perzepronen}, jedoch bestehen inzwischen einige weitere Architekturen, wobei das \textit{Deep Convolutional Generative Adversarial Network} (DCGAN) an beachtlicher Bedeutung gewonnen hat.
\cites[vgl.][]{4}[S. 5]{5}[S. 7]{6}

(Generator erstellt Inferenz)
\subsection{Generative und diskriminative Modelle}
Wie bereits angeführt, setzt sich ein GAN aus einem \textit{generativen} und einem \textit{diskriminativen} Modell zusammen. Doch welchem Sinn und Zweck dienen diese Modelle konkret? Diese Frage soll nachstehend ausführlich beantwortet werden. \\

Obwohl beide Vorgehen mit echten und generierten Daten arbeiten, verfolgen sie unterschiedliche Absichten. 
Sei X eine Dateninstanz und Y zugehörige Label, so ergibt sich Tabelle \ref{tab:gen_dis_Modell}:

\begin{table}[h]
\centering

	\begin{tabular}[h]{p{2.8cm}|p{7.5cm}|p{4.5cm}}
		Modell & Wahrscheinlichkeit & Ziel 
		\rule{0pt}{1.5em}\\
		\hline
		\rule{0pt}{1.5em}
		\textit{Generatives} & bestimmt die Wahrscheinlichkeit P(X,Y), bzw. P(X), also Wahrscheinlichkeit eines konkreten Beispiels X & Erstellen von überzeugend realistischen Daten\\
		\hline
		\rule{0pt}{1.5em}
		\textit{Diskriminatives} & arbeitet mit der bedingte Wahrscheinlichkeit P(X|Y), also wie wahrscheinlich es ist, dass eine gewisse Instanz einem bestimmten Label zugeordnet werden kann & Differenzierung verschiedener Instanzen\\
	\end{tabular}
	\captionsetup{justification=centering}
	\caption{Aufgabe eines \textit{generativen} und \textit{diskriminativen} Modells \\ Daten in Anlehnung an \cite[vgl.][]{30}}
	\label{tab:gen_dis_Modell}
\end{table}

Das \textit{generative} Modell modelliert hierfür die Verteilung der Daten, die nachgebildet werden sollen, in einem Datenraum $\mathbb{R}^n$. Die Aufgabe ist nun, Instanzen zu erstellen, die möglichst dicht an den realen Werten liegen. Je geringer der Abstand zwischen den Daten, desto realistischer die Nachbildung.
Das \textit{diskriminative} Modell muss dabei eine Grenze im Datenraum bestimmen, welche Instanzen mit verschiedenen Bezeichnungen (engl. Label) voneinander separiert. Kann der Algorithmus eine optimale Abgrenzung dezidieren, können Daten zugeordnet werden, ohne Angabe der jeweiligen bedingten Wahrscheinlichkeiten. Es fungiert demnach als Klassifikator.\\

In Abbildung \ref{fig:gen_dis_Modell} bilden grüne Symbole echte und orangene generierte Daten ab. Dabei ist zu erkennen wie das \textit{generative} Modell die Verteilung modelliert und Instanzen bereits nahe der originalen Daten erstellt. Das \textit{diskriminative} Modell findet hier dennoch eine nahezu optimale Lösung. Lediglich ein Wert wurde fälschlicherweise als Echt klassifiziert.

\begin{figure}[H]
	\centering
	\begin{center}	
		\text{Generatives Modell  \hspace{3.7cm} Diskriminatives Modell}\\
		\vspace{0.5cm}
		\includegraphics[width=0.25\textwidth]{img/generatives_Modell.png}
		\hspace{3cm}
		\includegraphics[width=0.25\textwidth]{img/diskriminatives_Modell.png}
	\end{center}
	\caption{Generatives und diskriminatives Modell}
	\label{fig:gen_dis_Modell}
\end{figure}

Die Aufgabe eines \textit{generativen} Modells ist damit deutlich komplexer. Auch Ian J. Goodfellow verweist in seiner Publikation auf Schwierigkeiten der Berechnung dieses Modells. 

Ein \textit{Generative Adversarial Network} realisiert die großartige Idee, die beiden Modell konkurrieren zu lassen, wodurch ein gegenseitiges Trainieren entsteht. Davon profitiert primär das \textit{generative} Modell, denn das Feedback des \textit{diskriminativen} dient als Hilfestellung für die Modellierung neuer Daten.
So kann die Leistung ganz ohne menschlichen Eingriff sukzessive gesteigert werden, bis Instanzen entstehen, welche nicht länger von den Originalwerten differenziert werden können. 

\cite[vgl.][]{30}

\subsection{Definition eines Autoencoders und dessen Rolle in einem GAN}

Ein \textit{Autoencoder} ist ein \textit{unüberwachtes künstliches neuronales Netz}, welches Eingangsinformationen $x$ zu $z$ komprimiert, um daraus eine Nachbildung $x*$ zu erstellen.
Während $x$ und $x*$ dieselbe Dimension aufweisen, findet bei der Komprimierung zu $z$ eine Dimensionsreduktion statt, weshalb die Ausmaße von $z$ wesentlich geringer sind. \\

Da es sich hier um zwei Verfahren handeln, setzt sich dieses \textit{künstliche neuronale Netz} aus zwei Abschnitten zusammen: Der \textit{Kodierer} (engl. Encoder) und der \textit{Dekodierer} (engl. Decoder). 

Ersterer komprimiert die Informationen $x$. Die Aufgabe der Komprimierung ist es, relevante Informationen zu selektieren und nur diese weiterzuleiten. Dabei entsteht die Dimension $z$, welches damit eine Repräsentation verdichteter Daten ist. Dies befindet sich in einem sogenannten \textit{verborgenem Raum}, der in der KI-Welt als \textit{latent space} bekannt ist. $z$ beinhaltet demnach nur die wichtigsten Merkmale der Eingangsdaten. \cite{33}

Zweiterer nutzt die Werte des \textit{latent space}, um die Rekonstruktion $x*$ zu bilden. \\


So wird erreicht, dass nur wesentliche Informationen weitergeleitet werden.

Folgende Abbildung zeigt schematisch das Verfahren eines \textit{Autoencoders}. 

\begin{figure}[H]
	\centering
	\includegraphics[width=0.5\textwidth]{img/autoencoder.png}
	\caption{Autoencoder, eigene Darstellung in Anlehnung an \cite[S. 20]{5}}
	\label{fig:autoencoder}
\end{figure}

Dieses \textit{künstliche neuronale Netz} bringt einige Vorteile mit sich:

\begin{itemize}
	\item Da es sich um ein \textit{Netz} im Bereich des \textit{unüberwachten Lernens} handelt, muss im Vorfeld nicht abgeklärt werden, zu welchem Sinn und Zweck das Verfahren explizit genutzt wird. Das Netz lernt selbst anhand der Inputdaten und passt sich diesen an.
	\item Dadurch ist eine konkrete Kennzeichnung der Eingabedaten überflüssig.
	\item Es müssen lediglich die komprimierten Daten übertragen werden. Dadurch wird der Informationsfluss erhöht, trotz Verringerung des Datenumfangs.
	\item Eingangsdaten können ohne Rauschen reproduziert werden.
\end{itemize}

\glqq For generation, [...] cut off the encoder part and use only the latent space and the decoder.\grqq{} \cite[S. 24]{5}
Um bei dieser Methode dennoch eine realitätsnahe Nachbildung ermöglichen zu können, muss der \textit{latent space} näher betrachtet werden. In der Folge, dass es sich nicht länger um eine Komprimierung der Daten $x$ handelt - eine logische Konsequenz, da auf die Verwendung des \textit{Encoders} verzichtet wird - müssen ein adäquater Raum gewählt werden. Hierfür hat sich eine Gleichverteilung mit einer festen Standardabweichung und angepasstem Mittelwert bewährt, wie beispielsweise die Gauß-Verteilung.

Ein \textit{Generatives adversarial Netz} macht sich dieses Verfahren zu nutze und adaptiert das Prinzip des \textit{Decoders}, der Werte aus einem \textit{latent space} entnimmt. So entsteht der \textit{Generator} eines GANs. 
\cites[vgl.][]{31,32}[S. 18-25]{5}

\subsection{Struktur eines GANs}

Formal gesprochen besteht ein \textit{Generative adversarial Network} aus zwei differenzierbaren Funktionen mit individueller Fehlerfunktion, die durch \textit{künstliche neuronale Netze} mit zugehörigen Modellparametern $\Theta$ repräsentiert werden. Damit sind die Netzwerke mittels \textit{Backpropagation} zu optimieren. \\
Während der \textit{Diskriminator} die Fehlinterpretation von echten und gefälschten Daten minimieren möchte, versucht der \textit{Generator} die Interpretation des \textit{Diskriminator} in Bezug auf die gefälschten Beispiele zu maximieren. 
\cite[vgl.][S. 37]{5} \cite[vgl.][S. 2]{4}

\begin{figure}[H]
	\centering
	\includegraphics[width=0.8\textwidth]{img/Architektur_GAN.png}
	\caption{Architektur eines GANs, eigene Darstellung in Anlehnung an \cite[S. 37]{5}}
	\label{fig:architektur}
\end{figure}

Das Zusammenspiel der einzelnen Komponenten ist in Abbildung \ref{fig:architektur} zusammengefasst. Um die Architektur zu verstehen, werden nachfolgend die einzelnen Bestandteile detailliert betrachtet:

\begin{enumerate}
	\item Im Falle eines GANs handelt es sich bei dem \textbf{latent space} typischerweise um eine Gauß-Verteilung mit einer Standardabweichung von eins und einem Mittel bei 0. Diese enthält zu Beginn jedoch keine ausschlaggebenden Informationen. 
	Werte, die aus dieser Verteilung entnommenen werden, ergeben einen \textit{random noise vector} $z$ -  vorwiegend ein Vektor mit 100 Einträgen - welcher als Input des \textit{Generators} dient.
	\cite[vgl.][S. 18, S. 37]{5} \cite[vgl.][]{34}
	
	\item Das Ziel des \textbf{Generators} ist es, eine Datenverteilung $p\textsubscript{g}$ zu lernen, die mit der Verteilung der Trainingsdaten $p\textsubscript{x}$ konvergiert. Die Funktion $G$, die durch den \textit{Generator} repräsentiert wird, wird durch die Modellparameter $\Theta\textsubscript{g}$ bestimmt. Diese erhält als Input einen \textit{random noise vector} $z$. Wird die Funktion $G$ nun auf $z$ angewandt, entsteht ein generierter Output $G(z)$. 
	Während des Trainings lernt der \textit{Generator}, Werte die einen bestimmten Output ergeben, in dem \textit{latent space} zuzuordnen \cite{34}.\cite[vgl.][S.5, S. 37]{5}
	
	\item Die Beispiele eines \textbf{Datensatzes}, mit welchem das \textit{Generative adversarial Network} trainiert wird, sind diese, deren Verteilungen rekonstruiert werden sollen. Damit bestimmt ein Datensatz, welche Verteilung gelernt wird. \cite[vgl.][S. 7]{5}

	\item Der \textbf{Diskriminator} nimmt als Input ein echtes $x$ oder ein gefälschtes $G(z)$ Beispiel entgegen. Die Aufgabe besteht darin, zu lernen, diese richtig zu klassifizieren. Dabei gibt die Funktion $D$, welche durch den \textit{Diskriminator} vertreten und durch die Modellparameter $\Theta\textsubscript{d}$ bestimmt ist, einen Skalar $D(x)$ aus. Dieser gibt für jeden Input die geschätzte Wahrscheinlichkeit (Wert zwischen [0,1]), dass $x$ ein echtes Beispiel ist.  \cite[vgl.][S. 1f.]{4}

	\item Die \textbf{Fehlerrückführung} basiert auf der Ausgabe $D(x)$ des \textit{Diskriminators}. Dabei variiert die Notation: $D(x)$ wird verwendet, wenn es sich bei einem Input um ein echtes, $D(x^*)$ oder $D(G(z))$, wobei $D(x^*) = D(G(z))$ ist, wenn es sich um generiertes Beispiel handelt. Werden die genannten Ziele der \textit{Netze} betrachtet, ergibt sich daraus folgende Tabelle:
	
	\begin{table}[H]
		\centering
		\begin{tabulary}{20cm}{L|L|L}
			\multirow{2}*{Input} & \multicolumn{2}{c}{zu erreichende Ausgabe} 
			\rule{0pt}{1.5em}\\
				
			\rule{0pt}{1.5em}
			&  Diskriminator  &  Generator \\
			\hline
			\rule{0pt}{1.5em}
			\textbf{Echt} ($x$) & 1 & / \\
			\rule{0pt}{1.5em}
			\textbf{Generiert} ($G(z)$) & 0 & 1 
			\rule{0pt}{1.5em}
		\end{tabulary}
		\captionsetup{justification=centering}
		\caption{Der Wert der Ausgabe des Diskriminators, die das jeweilige Netz versucht zu 		erreichen}
		\label{tab:Ziel_Ausgabe}
	\end{table}
	
	damit ergibt sich folgende Terminologie, welche Tabelle \ref{tab:Dis_Output} vorlegt:\\
	\textit{True positive}: Echte Beispiele korrekt als solche klassifiziert \\
	\textit{False negative}: Echt Beispiele fälschlicherweise als generierte klassifiziert \\
	\textit{True negative}: Generierte Beispiele korrekt als solche klassifiziert \\
	\textit{False positive}: Generierte Beispiele fälschlicherweise als echte klassifiziert \\
	\cite[vgl.][S. 41]{5}
	
	
	\begin{table}[H]
		\centering
		\begin{tabulary}{20cm}{L|L|L}
			\multirow{2}*{Input} & \multicolumn{2}{c}{Ausgabe des Diskriminators} 
			\rule{0pt}{1.5em}\\
			
			\rule{0pt}{1.5em}
			&  \textasciitilde 1  &  \textasciitilde 0 \\
			\hline
			\rule{0pt}{1.5em}
			\textbf{Echt} ($x$) & True positive & False negative \\
			\rule{0pt}{1.5em}
			\textbf{Generiert} ($G(z)$) & False positive & True negative 
			\rule{0pt}{1.5em}
		\end{tabulary}
		\captionsetup{justification=centering}
		\caption{Ausgabe eines Diskriminators, Daten in Anlehnung an \cite[vgl.][S. 41]{5}}
		\label{tab:Dis_Output}
	\end{table}

\end{enumerate}

	%hier ID 5 S. 37- 41
\subsection{Der Lernprozess eines GANs}

	\subsubsection{Die Fehlerfunktionen} \label{loss-function}
	
	\subsubsection{Sequenz des Generators}
	
	\subsubsection{Sequenz des Diskriminators}
	
	\subsubsection{Herausforderungen des Lernprozesses eines GANS}

\subsection{Herausforderungen des Lernprozesses}
%auch hier zeigen sich dieselben Probleme wie in Kapitel \ref{HerausforderungenDesLernens}. Doch hier kommen noch spezielle hinzu:
\textbf{konvergiert nicht}
\textbf{Mode collapse}
%Auch die Wahl der HP muss hier nochmal eingehend betrachtet werden. Ein netz kann unterschiedlich groß sein, die Batchgröße ist wichtig, nehm ich einen Bias hinzu...?
\textbf{Wahl der Hyperparameter}
\subsection{Varianten des GANs}
%Die kleinen wie Image-to-Image and so on
%BigGAN
%WassersteinGAN
\subsection{Das Deep Convolutional Generative Adversarial Network}
%Ist auch eine Variante, aber der Code, auf welchem diese Arbeit beruht, ist ein DCGAN. Daher soll zunächst die Theorie diskutiert werden, um anschließend Codeentscheidungen zu belegen
\end{document}
\newpage

\section{Neuartige Netzwerkarchitekturen}

Ziel der neuartigen Architektur: besseres/einfacheres Training. Weniger Probleme, effizientere Bildgenerierung
\subsection{Die Idee}


\subsection{Die Architektur}
\subsubsection{Architektur1}
\subsubsection{Architektur2}

\subsection{Umsetzung der Architektur}
\subsubsection{Architektur1}
\subsubsection{Architektur2}

\subsection{Probleme?}

\newpage

\section{Evaluation der neuartigen Netzwerkarchitekturen}

\subsection{Ergebnisse}

\subsection{Probleme}


\newpage

\section{Fazit und Ausblick}

Das entstandene Vorgehen ist ziemlich krass
\newpage

\pagenumbering{Roman}
\setcounter{page}{7}


\begin{document}
	

\section{Anhang}
%\addcontentsline{toc}{section}{Anhang}
%\markboth{\MakeUppercase {Anhang}}{}

% \input{VI_Abbildungsverzeichnis}
%\addcontentsline{toc}{subsection}{Abbildungsverzeichnis}
%\listoffigures
%\newpage

% \input{VII_Tabellenverzeichnis}
%\addcontentsline{toc}{subsection}{Tabellenverzeichnis}
%\listoftables
%\newpage	

%\begin{figure}



\subsection{Abbildungsverzeichnis}
%\makeatletter
%\@starttoc{lof}% Print List of Figures
%\makeatother
%\addcontentsline{toc}{subsection}{3.1 Abbildungsverzeichnis}
\listoffigures
\newpage

\subsection{Tabellenverzeichnis}
%\makeatletter
%\@starttoc{lot}% Print List of Tables
%\makeatother 
%\addcontentsline{toc}{subsection}{Tabellenverzeichnis}
\listoftables
\newpage

%\printbibliography[notkeyword=Quelle,title={\subsection*{Literaturverzeichnis}},heading=subbibliography]
\subsection{Literaturverzeichnis}
%\addcontentsline{toc}{subsection}{Literaturverzeichnis}
\printbibliography[heading=none]
%\clearpage


%\subsection*{Erklärung und Ermächtigung}
%%\section{Erklärung und Ermächtigung}
%\label{Erklärung und Ermächtigung}

%\begin{document}


\subsection{Erklärung}



Ich versichere, dass ich diese Abschlussarbeit selbstständig angefertigt, nicht anderweitig für Prüfungszwecke vorgelegt, alle benutzten Quellen und Hilfsmittel angegeben, sowie wörtliche und sinngemäße Zitate gekennzeichnet habe.
\\[2.5cm]

Kempten, den:  \hrulefill\enspace\enspace\enspace  Unterschrift:  \hrulefill
\\[2.5cm]

%\end{document}

\subsection{Ermächtigung}
\label{Ermächtigung}
%\begin{document}

Hiermit ermächtige ich die Hochschule Kempten und den betreuenden Dozenten, Herrn Prof. Dr. Jürgen Brauer, zur Veröffentlichung der Kurzzusammenfassung (Abstract) meiner Arbeit (im Wortlaut wie sie in der Arbeit enthalten ist) - einschließlich der Angabe meines Namens, des Titels der Arbeit und der Name der Firma, bei der die Arbeit durchgeführt wurde - auf gedruckten Medien oder auf einer Internetseite (z.B. im Rahmen einer Auflistung der durch den Dozenten betreuten studentischen Abschlussarbeiten).
\\[2.5cm]

Kempten, den:  \hrulefill\enspace\enspace  Unterschrift:  \hrulefill

%\end{document}
%\addcontentsline{toc}{subsection}{Erklärung und Ermächtigung}
\newpage
\subsection{Erklärung}



Ich versichere, dass ich diese Abschlussarbeit selbstständig angefertigt, nicht anderweitig für Prüfungszwecke vorgelegt, alle benutzten Quellen und Hilfsmittel angegeben, sowie wörtliche und sinngemäße Zitate gekennzeichnet habe.
\\[2.5cm]

Kempten, den:  \hrulefill\enspace\enspace\enspace  Unterschrift:  \hrulefill
\\[2.5cm]

%\end{document}

\subsection{Ermächtigung}
\label{Ermächtigung}
%\begin{document}

Hiermit ermächtige ich die Hochschule Kempten und den betreuenden Dozenten, Herrn Prof. Dr. Jürgen Brauer, zur Veröffentlichung der Kurzzusammenfassung (Abstract) meiner Arbeit (im Wortlaut wie sie in der Arbeit enthalten ist) - einschließlich der Angabe meines Namens, des Titels der Arbeit und der Name der Firma, bei der die Arbeit durchgeführt wurde - auf gedruckten Medien oder auf einer Internetseite (z.B. im Rahmen einer Auflistung der durch den Dozenten betreuten studentischen Abschlussarbeiten).
\\[2.5cm]

Kempten, den:  \hrulefill\enspace\enspace  Unterschrift:  \hrulefill

%\end{figure}
\end{document}

%\addcontentsline{toc}{section}{Anhang}
\newpage


\begin{document}

\newpage
\thispagestyle{empty}
\quad  \addtocounter{page}{-1}
\newpage

\end{document}
%\input{VII_Tabellenverzeichnis}
%\addcontentsline{toc}{section}{Tabellenverzeichnis}
%\newpage


%\printbibliography[notkeyword=Quelle,title={\section*{Literatur}},heading=subbibliography]
%\addcontentsline{toc}{section}{Literaturverzeichnis}
%\newpage

%
\begin{document}
	

\section{Anhang}
%\addcontentsline{toc}{section}{Anhang}
%\markboth{\MakeUppercase {Anhang}}{}

% \input{VI_Abbildungsverzeichnis}
%\addcontentsline{toc}{subsection}{Abbildungsverzeichnis}
%\listoffigures
%\newpage

% \input{VII_Tabellenverzeichnis}
%\addcontentsline{toc}{subsection}{Tabellenverzeichnis}
%\listoftables
%\newpage	

%\begin{figure}



\subsection{Abbildungsverzeichnis}
%\makeatletter
%\@starttoc{lof}% Print List of Figures
%\makeatother
%\addcontentsline{toc}{subsection}{3.1 Abbildungsverzeichnis}
\listoffigures
\newpage

\subsection{Tabellenverzeichnis}
%\makeatletter
%\@starttoc{lot}% Print List of Tables
%\makeatother 
%\addcontentsline{toc}{subsection}{Tabellenverzeichnis}
\listoftables
\newpage

%\printbibliography[notkeyword=Quelle,title={\subsection*{Literaturverzeichnis}},heading=subbibliography]
\subsection{Literaturverzeichnis}
%\addcontentsline{toc}{subsection}{Literaturverzeichnis}
\printbibliography[heading=none]
%\clearpage


%\subsection*{Erklärung und Ermächtigung}
%%\section{Erklärung und Ermächtigung}
%\label{Erklärung und Ermächtigung}

%\begin{document}


\subsection{Erklärung}



Ich versichere, dass ich diese Abschlussarbeit selbstständig angefertigt, nicht anderweitig für Prüfungszwecke vorgelegt, alle benutzten Quellen und Hilfsmittel angegeben, sowie wörtliche und sinngemäße Zitate gekennzeichnet habe.
\\[2.5cm]

Kempten, den:  \hrulefill\enspace\enspace\enspace  Unterschrift:  \hrulefill
\\[2.5cm]

%\end{document}

\subsection{Ermächtigung}
\label{Ermächtigung}
%\begin{document}

Hiermit ermächtige ich die Hochschule Kempten und den betreuenden Dozenten, Herrn Prof. Dr. Jürgen Brauer, zur Veröffentlichung der Kurzzusammenfassung (Abstract) meiner Arbeit (im Wortlaut wie sie in der Arbeit enthalten ist) - einschließlich der Angabe meines Namens, des Titels der Arbeit und der Name der Firma, bei der die Arbeit durchgeführt wurde - auf gedruckten Medien oder auf einer Internetseite (z.B. im Rahmen einer Auflistung der durch den Dozenten betreuten studentischen Abschlussarbeiten).
\\[2.5cm]

Kempten, den:  \hrulefill\enspace\enspace  Unterschrift:  \hrulefill

%\end{document}
%\addcontentsline{toc}{subsection}{Erklärung und Ermächtigung}
\newpage
\subsection{Erklärung}



Ich versichere, dass ich diese Abschlussarbeit selbstständig angefertigt, nicht anderweitig für Prüfungszwecke vorgelegt, alle benutzten Quellen und Hilfsmittel angegeben, sowie wörtliche und sinngemäße Zitate gekennzeichnet habe.
\\[2.5cm]

Kempten, den:  \hrulefill\enspace\enspace\enspace  Unterschrift:  \hrulefill
\\[2.5cm]

%\end{document}

\subsection{Ermächtigung}
\label{Ermächtigung}
%\begin{document}

Hiermit ermächtige ich die Hochschule Kempten und den betreuenden Dozenten, Herrn Prof. Dr. Jürgen Brauer, zur Veröffentlichung der Kurzzusammenfassung (Abstract) meiner Arbeit (im Wortlaut wie sie in der Arbeit enthalten ist) - einschließlich der Angabe meines Namens, des Titels der Arbeit und der Name der Firma, bei der die Arbeit durchgeführt wurde - auf gedruckten Medien oder auf einer Internetseite (z.B. im Rahmen einer Auflistung der durch den Dozenten betreuten studentischen Abschlussarbeiten).
\\[2.5cm]

Kempten, den:  \hrulefill\enspace\enspace  Unterschrift:  \hrulefill

%\end{figure}
\end{document}

%\addcontentsline{toc}{section}{Anhang}
%\newpage

%%\section{Erklärung und Ermächtigung}
%\label{Erklärung und Ermächtigung}

%\begin{document}


\subsection{Erklärung}



Ich versichere, dass ich diese Abschlussarbeit selbstständig angefertigt, nicht anderweitig für Prüfungszwecke vorgelegt, alle benutzten Quellen und Hilfsmittel angegeben, sowie wörtliche und sinngemäße Zitate gekennzeichnet habe.
\\[2.5cm]

Kempten, den:  \hrulefill\enspace\enspace\enspace  Unterschrift:  \hrulefill
\\[2.5cm]

%\end{document}

\subsection{Ermächtigung}
\label{Ermächtigung}
%\begin{document}

Hiermit ermächtige ich die Hochschule Kempten und den betreuenden Dozenten, Herrn Prof. Dr. Jürgen Brauer, zur Veröffentlichung der Kurzzusammenfassung (Abstract) meiner Arbeit (im Wortlaut wie sie in der Arbeit enthalten ist) - einschließlich der Angabe meines Namens, des Titels der Arbeit und der Name der Firma, bei der die Arbeit durchgeführt wurde - auf gedruckten Medien oder auf einer Internetseite (z.B. im Rahmen einer Auflistung der durch den Dozenten betreuten studentischen Abschlussarbeiten).
\\[2.5cm]

Kempten, den:  \hrulefill\enspace\enspace  Unterschrift:  \hrulefill

%\end{document}
%\addcontentsline{toc}{section}{Erklärung und Ermächtigung}
%\newpage


%\thispagestyle{empty}
%\newpage
%\section*{}
%\newpage

\end{document}