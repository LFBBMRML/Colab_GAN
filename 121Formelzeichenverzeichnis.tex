\documentclass[12pt]{article}
\documentclass{standalone}
\usepackage[usenames]{color} % Farbunterstützung
\usepackage{amssymb}	% Mathe
\usepackage{amsmath} % Mathe
\usepackage[utf8]{inputenc} % Direkte Eingabe von Umlauten und anderen Diakritika
\begin{document}
\addsec{Formelzeichenverzeichnis}

\begin{longtable}{p{.20\textwidth}p{.20\textwidth}p{.50\textwidth}}
	\textbf{Formelzeichen}&\textbf{Einheit}&\textbf{Beschreibung}\\ \endhead
	\(A\) & \(m^2\) & Fläche \\
	\(B\) & \(T\) & Magnetische Flussdichte \\
	\(C\) & \(F\) & Elektrische Kapazität eines Kondensators \\
	\(D\) & \(m\) & Durchmesser der Spule \\
	\(E_i\) & \(\frac{V}{m}\) & Elektrische Feldstärke des induzierten Felds \\
	\(H\) & \(\frac{A}{m}\) & Magnetische Feldstärke \\
	\(L\) & \(H\) & Induktivität \\
	\(I\) & \(A\) & Stromstärke \\
	\(l\)& \(m\) & Länge der Spule \\
	\(N\) & \(1\) & Anzahl der Spulenwindungen \\
	\(n\) & \(1\) & Anzahl der Spulenwindungen \\
	\(R_{2}\)& \(\Omega\) & Elektrischer Widerstand von \(R_{2}\) \\
	\(R_{L}\) & \(\Omega\) & Elektrischer Widerstand des Lastwiderstands \\
	\(R_{S}\) & \(\Omega\) & Elektrischer Widerstand des Shunts \\
	\(Q\) &  \(1\) &  Schwingkreisgüte \\
	\(\mu\) &  \(\frac{H}{m}\) &  Magnetische Permeabilität \\
	\(\mu_{0}\) &  \(\frac{N}{A^2}\) &  Magnetische Feldkonstante \\
	\(\mu_{r}\) &  \(1\) &  Permeabilitätszahl \\
	\(\phi\) &  \(Wb\) &  Magnetischer Fluss \\
\end{longtable}
%\]
\end{document}